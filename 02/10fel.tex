\feladat{10}{
 Végtelen földelt vezető síktól $d$ távolságra $Q$ ponttöltés helyezkedik el. Ha\-tá\-roz\-zuk meg a síkon a felületi töltéssűrűséget!
}{}{}
 
\ifdefined\megoldas

 Megoldás: 

 Az előző feladatban kiszámoltuk a térerősséget a $z=0$ síkban:
 \al{
  \Ev_{z>0}(x,y,z=0)
   &= \left(0,0,\frac{-2kQd}{(x^2+y^2+d^2)^\frac{3}{2}})\right)\;.
 }

 Ahogy azt a 9.\ feladatban láttuk, a fém felületén a térerősség és a felületi töltéssűrűség között az alábbi kapcsolat áll fenn:
 \al{
  E_\perp=\frac{\sigma}{\ep_0}\;,
 }
 ahol $E_\perp$ a fém felületére merőleges térerősség nagysága és $\sigma$ a felületi töltéssűrűség. Ezt felhasználva:
 \begin{align}
  \sigma(x,y) &= \ep_0E_\perp(x,y,z=0) = \frac{-2kQ\ep_0 d}{(x^2+y^2+d^2)^\frac{3}{2}}\;.
 \end{align}

% \al{
%  \sigma(x,y)
%   &=\ep_0\Ev_{z>0}(x,y,z=0)
%    =\left(0,0,\frac{-2kQ\ep_0 d}{(x^2+y^2+d^2)^\frac{3}{2}})\right)\;.
% }

 Láthatjuk, hogy a töltéssűrűség $Q$-val ellentétes előjelű. Ez érthető, hiszen $Q$ hatására az ellentétes előjelű töltések halmozódnak fel a fém felületén. 

\fi