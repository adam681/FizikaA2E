\feladat{2}{
 Derékszögű koordinátarendszer $x$ tengelye mentén két töltés található. A $q_{1}= 7\me{\mu C}$ töltés az origóban helyezkedik el, míg a $q_{2}=-5\me{\mu C}$ töltés az origótól $d=0,3\me{m}$ távolságra van.
 \begin{enumerate}[label=\alph*),itemsep=0pt]
  \item Határozza meg az elektromos térerősséget az $y$ tengely mentén!
  \item Határozza meg az elektromos térerősséget az $x$ tengely mentén!
  \item Van-e a tengelyek mentén olyan pont, ahol a térerősség nulla?
  \item Van-e egyéb olyan pont a síkon, ahol a térerősség nulla?
  \item Hogyan változik a helyzet, ha $q_{2} = 5\me{\mu C}$ a második töltés?
 \end{enumerate}
}{}{}
 
\ifdefined\megoldas

 Megoldás: 
 \marginfigure{02fel_01fig.tikz}

 \begin{enumerate}[label=\alph*),itemsep=0pt]
  \item 
   Az $y$ tengely pozitív felén az 1-gyes töltés által kifejtett erő a pozitív irányba mutat, míg a másik töltés terének van $x$ és $y$ irányú komponense is. A térerősség ennek a két járuléknak az összege.
   \al{
    \Ev_1^{(+)}(y)
     &=k\frac{\abs{q_1}}{y^2}\left(0,1,0\right)
    \\
    \Ev_2^{(+)}(y)
     &=k\frac{\abs{q_2}}{y^2+d^2}\left(\sin\alpha,-\cos\alpha,0\right)
    \\
     &=k\frac{\abs{q_2}}{y^2+d^2}\left(\frac{d}{\sqrt{y^2+d^2}},-\frac{y}{\sqrt{y^2+d^2}},0\right)\;,
   }
   vagyis
   \al{
    \Ev^{(+)}(y)
     =k\left(\frac{\abs{q_2} d}{(y^2+d^2)^\frac{3}{2}},\frac{\abs{q_1}}{y^2}-\frac{\abs{q_2} y}{(y^2+d^2)^\frac{3}{2}},0\right)
   } 
   A negatív féltengelyen az $\Ev_1$ megfordul, vagyis lefele fog mutatni, illetve az $\Ev_2$ $y$ irányú járuléka pedig felfele fog mutatni, így 
   \al{
    \Ev^{(-)}(y)
     =k\left(\frac{\abs{q_2}d}{(y^2+d^2)^\frac{3}{2}},-\frac{\abs{q_1}}{y^2}-\frac{\abs{q_2} y}{(y^2+d^2)^\frac{3}{2}},0\right)\;.
   } 
   
  \item
   Az $x$ tengely mentén:
   \al{
    \Ev^{(++)}(x)
     &=k\left(\frac{\abs{q_1}}{x^2}-\frac{\abs{q_2}}{(x-d)^2},0,0\right)\;,\\
    \Ev^{(+-)}(x)
     &=k\left(\frac{\abs{q_1}}{x^2}+\frac{\abs{q_2}}{(x-d)^2},0,0\right)\;,\\
    \Ev^{(--)}(x)
     &=k\left(-\frac{\abs{q_1}}{x^2}+\frac{\abs{q_2}}{(x-d)^2},0,0\right)\;,
   }
   ahol a felső indexek az ábrán szereplő tartományoknak felelnek meg. 
   
  \item
   
   Az $y$ tengelyen mentén biztos, hogy nincs, mert ott a térerősségnek mindig van $x$ komponense. Az $x$ tengely mentén pedig van ilyen pont, meg kell keresni az $E(x)=0$ egyenlet megoldását. A $q_1$ és a $q_2$ arányától függően vagy a $(++)$, vagy a $(--)$ tartományon lesz megoldás.
   
  \item 
   Nincs. Csak a távoli végtelenben. 
   
  \item 
   Ekkor sem lesz stabil pont az $y$ tengely mentén. Az $x$ tengely mentén ekkor a stabil hely a két töltés között lesz. 
 \end{enumerate}

\fi