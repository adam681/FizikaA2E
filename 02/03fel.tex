\feladat{3}{
 A $q_{1}$ és $q_{2}$ pozitív töltéseket az $\rv_{1}$ és $\rv_{2}$ helyvektorokkal megadott pontokban rögzítjük. Az $\rv_{3}$ helyvektorú pontba $q_{3}$ töltést helyezve mindhárom töltésre zérus eredő erő hat. Határozzuk meg $q_{3}$-at és $\rv_{3}$-at!
}{}{}
 
\ifdefined\megoldas
 
 Megoldás: 

 \marginfigure{03fel_01fig.tikz}
 Értelemszerűen a harmadik töltésnek negatívnak kell lennie, illetve ennek a háromnak egy egyenesre kell esnie. 

 Legyen a $q_3$-as töltés az 1-gyes töltéstől $x$ a 2-es töltéstől pedig $y$ távolságra. Ekkor $\rv_3$ felírható úgy mint $\rv_3=\frac{x}{x+y}\rv_1+\frac{y}{x+y}\rv_2$.

 Azt tudjuk, hogy ekkor egyenként a három töltésre az erők eredője nulla:
 \al{
  0&=\Fv_{12}+\Fv_{13} &
  0&=\Fv_{21}+\Fv_{23} &
  0&=\Fv_{31}+\Fv_{32}\;,
 }
 és mivel az erők mind egy egyenesbe esnek, így a nagyságukra felírhatjuk, hogy
 \al{
  0&=F_{12}+F_{13} &
  0&=F_{21}+F_{23} &
  0&=F_{31}+F_{32}\;.
 }
 Ezek részletesen kiírva (majd $k$-val egyszerűsítve)
 \al{
  0&=\frac{q_1 q_3}{x^2}-\frac{q_1 q_2}{(x+y)^2} &
  0&=\frac{q_1 q_2}{(x+y)^2}-\frac{q_2 q_3}{y^2} &
  0&=\frac{q_1 q_3}{x^2}-\frac{q_2 q_3}{y^2}\;.
 }

 Az egyenletek átalakítva:
 \al{
  \frac{x}{x+y}&=\sqrt{\frac{q_3}{q_2}}&
  \frac{y}{x+y}&=\sqrt{\frac{q_3}{q_1}}&
  \frac{x}{y}&=\sqrt{\frac{q_1}{q_2}}\;,
 }
 majd az elsőt és a harmadikat felhasználva
 \al{
  \frac{x}{x+\sqrt{\frac{q_2}{q_1}}x}&=\sqrt{\frac{q_3}{q_2}}\\
  1&=\sqrt{\frac{q_3}{q_2}}+\sqrt{\frac{q_3}{q_1}}\\
  1&=\frac{q_3}{q_2}+\frac{q_3}{q_1}+2\frac{q_3}{\sqrt{q_1 q_2}}\\
  q_3&=\frac{q_1 q_2}{\big(\sqrt{q_1}+\sqrt{q_2}\big)^2}\;.
 }
 Innen pedig már egyszerűen következik, hogy 
 \al{
  \rv_3
   &=\sqrt{\frac{q_3}{q_2}}\rv_1+\sqrt{\frac{q_3}{q_1}}\rv_2
    =\frac{\sqrt{q_1}}{\sqrt{q_1}+\sqrt{q_2}}\rv_1+\frac{\sqrt{q_2}}{\sqrt{q_1}+\sqrt{q_2}}\rv_2
 }

\fi