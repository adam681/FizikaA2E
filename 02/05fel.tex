\feladat{5}{
 Két egyenlő sugarú fémgolyócskát $l=1$ méter hosszú fonálon közös pontban felfüggesztünk. Mindkét golyóval azonos $q$ nagyságú töltést közlünk. Az elektromos taszítás hatására $d=20\me{cm}$-re eltávolodnak egymástól. Mekkora a golyók töltése, ha tömegük $m=1\me{g}$?
}{05fel_01fig.tikz}{}
 
\ifdefined\megoldas

 Megoldás: 

 Mivel a golyók nyugalomban vannak, így a rájuk ható erők eredője nulla. Ez felírva az $x$ és $y$ irányú komponensekre:
 \al{
  x: && 0&=F_\text{C}-K\sin\alpha&
  y: && 0&=-mg+K\cos\alpha\;.
 }
 A geometria alapján a bezárt szög:
\al{
 \sin\alpha &= \frac{d/2}{l}
 &\Rightarrow&&
 \tg\alpha
  &=\frac{\sin\alpha}{\sqrt{1-\sin^2\alpha}}
   =\frac{d}{\sqrt{4l^2-d^2}}\;,
}
és az egyensúlyi egyenletekből
 \al{
  F_\text{C}
   &=k\frac{q^2}{d^2}
    =mg\tg\alpha
  \\
  q
   &=d\sqrt{\frac{mg}{k}\tg\alpha}
    =0,2\me{m} \sqrt{\frac{10^{-3}\me{kg}\cdot 10\me{\frac{m}{s^2}}}{9\cdot 10^{9}\me{\frac{N\,m^2}{C^2}}}\cdot \frac{0,2\me{m}}{\sqrt{4\cdot(1\me{m})^2-(0,2\me{m})^2}}}
  \\
   &=6,68\cdot 10^{-8}\me{C}\;.
 }
 
\fi