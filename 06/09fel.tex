\feladat{9}{
 Az $R_1$ és $R_2$ sugarú koncentrikus gömbök közötti térrészt inhomogén szigetelő tölti ki, amelynek permittivitása a középponttól mért távolság függvénye. Milyen függvény szerint kell változnia a permittivitásnak, hogy a kondenzátort feltöltve az elektromos térerősség nagysága az egész térrészben állandó legyen? Mekkora az így kapott kondenzátor kapacitása?
}{09fel_01fig}{}

\ifdefined\megoldas

 Megoldás: 

 Azt tudjuk, hogy ilyen esetben a dielektromos eltolás a gömbök között:
 \al{
  D(r)
   =\frac{1}{4\pi}\frac{Q}{r^2}\;,
 }
 % \marginfigure{}
 ha a kondenzátort $Q$ töltéssel töltjük fel. A térerősség ebből: $E(r)=\frac{D(r)}{\ep_0\ep_\text{r}(r)}$, vagyis ahhoz, hogy $E(r)$ ne függjön a sugártól, $\ep_\text{r}(r)\sim\frac{1}{r^2}$ kell hogy legyen. Legyen akkor 
 \al{
  \ep_\text{r}(r)
   &=\frac{\tilde\ep}{r^2}
  &&\Rightarrow&
  E(r)
   &=\frac{1}{4\pi\ep_0\tilde\ep}\cdot Q
  \;.
 }

 A feszültség a két gömbfelület között:
 \al{
  V
   &=-\intl{R_2}{R_1}E(r')\,\dd r'
    =-\frac{1}{4\pi\ep_0\tilde\ep}\cdot Q\intl{R_2}{R_1}\,\dd r'
    =\frac{1}{4\pi\ep_0\tilde\ep}\cdot Q (R_2-R_1)\;,
 }
 vagyis a kapacitás:
 \al{
  C=\frac{Q}{V}
   =\frac{4\pi\ep_0\tilde\ep }{R_2-R_1}\;.
 }

\fi