\feladat{5}{
 Koncentrikus fémgömbök között $R < r < 2R$ tartományban $\varepsilon_{1}$ permittivitású, $E_{1kr}$ átütési szilárdságú, a $2R < r < 3R$ tartományban pedig $\varepsilon_{2}=0,25\varepsilon_{1}$ permittivitású, $E_{2kr}=1,1E_{1kr}$ átütési szilárdságú szigetelő van. Mekkora a gömbökre kapcsolható legnagyobb feszültség?
}{}{}

\ifdefined\megoldas

 Megoldás: 
 
 \marginfigure{05fel_01fig}

 Korábbi tanulmányainkból ismerjük már, hogy hogyan lehet kiszámolni a dielektromos eltolást és a térerősséget. Ezek az adott rendszerben:
 \al{
  D(r)&=
   \begin{cases}
    0 & r<R_1 \\
    \frac{1}{4\pi}\frac{Q}{r^2} & 2R<r<3R \\
    0 & 3R<r
   \end{cases}
  &,&&
  E(r)&=
   \begin{cases}
    0 & r<R \\
    \frac{1}{4\pi\ep_0 \ep_1}\frac{Q}{r^2} & R<r<2R \\
    \frac{1}{4\pi\ep_0 \ep_2}\frac{Q}{r^2} & 2R<r<3R \\
    0 & 3R<r
   \end{cases}
   \;.
 }

 Láthatjuk, hogy mind a két tartományban a legkisebb sugárnál a legnagyobb a térerősség, vagyis akkor nem üt át a rendszer, ha $E(R)<E_\text{1kr}$ és $E(2R)<E_\text{2kr}$. A feladat számértékei úgy vannak megadva, hogy $E(R)=E(2R)$, vagyis elégséges, ha 
 \al{
  E(R)&<E_\text{1kr} \\
  \frac{1}{4\pi\ep_0 \ep_1}\frac{Q}{R^2} &< E_\text{1kr} \\
  Q &< 4\pi\ep_0 \ep_1 R^2 E_\text{1kr}
  \;.
 }

 Most már tudjuk, hogy legfeljebb mekkora lehet az a töltés, amit a kondenzátorra felhalmozhatunk. Innen a legnagyobb feszültséget a kondenzátor kapacitásának ismeretében tudjuk megadni:
 \al{
  U=\frac{Q}{C}<\frac{1}{C}\cdot 4\pi\ep_0 \ep_1 R^2 E_\text{1kr}\;.
 }

 A kapacitás egyszerűen számolható, ha észrevesszük, hogy ez a kondenzátor olyan, mintha két kondenzátor lenne sorosan kapcsolva:
 \al{
  C=\frac{1}{\frac{1}{C_1}+\frac{1}{C_2}}\;,
 }
 ahol a $C_1$ az $R$ és $2R$ sugarú gömbök közötti $\ep_1$ dielektrikummal töltött kondenzátornak, illetve a $C_2$ az $2R$ és $3R$ sugarú gömbök közötti $\ep_2$ dielektrikummal töltött kondenzátor kapacitásának felel meg. Az előző feladat eredményét felhasználva:
 \al{
  C
   =\frac{1}{\frac{1}{\ep_1}\left(\frac{1}{R}-\frac{1}{2R}\right)+\frac{1}{4\pi\ep_0 \ep_2}\left(\frac{1}{2R}-\frac{1}{3R}\right)}
   =\frac{4\pi\ep_0 R}{\frac{1}{2\ep_1}+\frac{1}{6\ep_2}}\;,
 }
 vagyis
 \al{
  U
   &<
     \left[\frac{1}{4\pi\ep_0 R}\left(\frac{1}{2\ep_1}+\frac{1}{6\ep_2}\right)\right]\cdot 4\pi\ep_0 \ep_1 R^2 E_\text{1kr}
    =\left(\frac{1}{2}+\frac{1}{6}\frac{\ep_1}{\ep_2}\right)\cdot R E_\text{1kr}
  \;.
 }

\fi