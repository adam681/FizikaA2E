\feladat{4}{
 Oldjuk meg a 2. feladatot a csomóponti potenciálok módszerével!
}{04fel_01fig}{}

\ifdefined\megoldas
  
 Megoldás: 

 Ebben az áramkörben két független csomópont található, a harmadik potenciálját tudjuk rögzíteni. A huroktörvények az egyes csomópontokra:
 \al{
  0 &= \frac{U_A-40\me{V}}{2\me{\Omega}}+ \frac{U_A}{10\me{\Omega}}+  \frac{U_A-U_B}{9\me{\Omega}}
  \\
  0 &= \frac{U_B-U_A}{9\me{\Omega}}+ \frac{U_B}{4\me{\Omega}}-1\me{A}\;.
 }
 A két egyenletet bővítve:
 \al{
  0 &= 45 U_A-1800\me{V}+ 9 U_A+ 10 U_A- 10 U_B
  \\
  0 &= 4 U_B-4 U_A + 9 U_B-36\me{V}
 }
 majd összevonva
 \al{
  0 &= 64 U_A - 10 U_B - 1800\me{V}
  \\
  0 &=-4 U_A + 13 U_B - 36\me{V}\;.
 }
 Az elsőből $U_B$ kifejezve
 \al{
  U_B &= 6,4 U_A - 180\me{V}\;,
 } 
 majd a másodikba helyettesítve
 \al{
  0 &=-4 U_A + 13\cdot(6,4 U_A - 180\me{V})  - 36\me{V}
  \\
  79,2 U_A &=2376\me{V}
  \\
  U_A &=30\me{V}\;,
 }
 melyet felhasználva
 \al{
  U_B
   =6,4\cdot 30\me{V}-180\me{V}
   =12\me{V}\;.
 }
 A kérdezett $I_1$ áram: $I_1=\frac{U_A}{10\me{\Omega}}=3\me{A}$.

\fi