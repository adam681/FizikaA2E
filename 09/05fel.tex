\feladat{5}{
 Határozzuk meg az ábrán látható kapcsolás Th\'evenin-féle helyettesítő képét! Mekkora az eredeti kapcsolásban és a helyettesítő képben a generátorok teljesítménye?
}{05fel_01fig}{1}

\ifdefined\megoldas

 Megoldás: 

 A helyettesítő kép meghatározásához először számoljuk ki, hogy terhelés nélkül mekkora a feszültség az $A$ és a $B$ pontok között. Ehhez a huroktörvény felírva
 \al{
  0 &= -30\me{V} +\tilde I_1\cdot 12\me{\Omega} +\tilde I_1\cdot 12\me{\Omega} + 24\me{V} \\
  \tilde I_1 &= 0,25\me{A}\;.
 }
 Ahonnan
 \al{
  U_{AB}=-24\me{V}-0,25\me{A}\cdot 12\me{\Omega}=-27\me{V}\;.
 }

 \marginfigure{05fel_02fig}
 Ezen kívül számoljuk ki, hogy mekkora az $A$ és a $B$ pontok között folyó áram, ha azok között rövidzár található:
 \al{
  0 &= -30\me{V} +\tilde I_1\cdot 12\me{\Omega} +(\tilde I_1-\tilde I_2)\cdot 12\me{\Omega} + 24\me{V} \\
  0 &= - 24\me{V}+(\tilde I_2-\tilde I_1)\cdot 12\me{\Omega}\;,
 }
 ahol a második egyenletből
 \al{
  \tilde I_1-\tilde I_2 =-2\me{A}\;,
 }
 melyet az első egyenletbe behelyettesítve
 \al{
  0 &= -30\me{V} +\tilde I_1\cdot 12\me{\Omega}\\
  \tilde I_1 &= 2,5\me{A}\;,
 }
 illetve
 \al{
  \tilde I_2 &= 4,5\me{A}\;.
 }
 A rövidzáron így $4,5\me{A}$ folyik. 

 \marginfigure{05fel_03fig}
 A helyettesítő képben tehát a Th\'evenin-ellenállás
 \al{
  R_\text{Th}
   =\frac{U_\text{üres}}{I_\text{rövidzár}}
   =\frac{27\me{V}}{4,5\me{A}}
   =6\me{\Omega}\;,
 }
 illetve a feszültségforrás $U_\text{Th}=U_\text{üres}=27\me{V}$.

 Azt láthatjuk, hogy az eredeti áramkörben akkor is folyik áram, ha az nincs terhelve, vagyis ha az $A$--$B$ pontokra semmi nincs rákapcsolva. Ez az áram az ellenállásokon $P_\text{veszteség}=24\me{\Omega}\cdot (0,25\me{A})^2=1,5\me{W}$ veszteséget ad. A $30\me{V}$-os generátor $P_{30\me{V}}=30\me{V}\cdot 0,25\me{A}=7,5\me{W}$ teljesítményt ad le, míg a $24\me{V}$-os telepen $P_{24\me{V}}=24\me{V}\cdot 0,25\me{A}=6\me{W}$ teljesítmény esik. 

 Vegyük észre, hogy a $30\me{V}$-os telep itt leadja a teljesítményt, míg a $24\me{V}$-os telepen úgy folyik az áram, hogy az azt tölti, vagyis ez a telep $6\me{W}$-ot felvesz az áramkörből. Láthatjuk, hogy itt is teljesül az energiamegmaradás. 

 Ezzel szemben a helyettesítő képben csak terhelés jelenlétében folyik áram. Innen következik az, hogy az eredeti és a helyettesítő képben a teljesítményviszonyok nem feleltethetőek meg ilyen egyszerűen.
 
\fi