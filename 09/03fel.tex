\feladat{3}{
 Oldjuk meg az 1. feladatot a csomóponti potenciálok módszerével!
}{03fel_01fig}{}

\ifdefined\megoldas
  
 Megoldás: 

 Az áramkörben két csomópont található. Az egyiken rögzítsük a potenciált: a $B$ pontot leföldeljük. Ekkor az $A$ pont potenciálja lesz az egyetlen ismeretlen: $U_A$.

 Az egyetlen független $A$ csomópontra írjuk fel a csomóponti törvényt:
 \al{
  0&=\frac{U_A-32\me{V}}{2\me{\Omega}}+\frac{U_A}{8\Omega}+\frac{U_A-20\me{V}}{4\me{\Omega}}
  \\
  0&=4 U_A-128\me{V}+ U_A + 2 U_A-40\me{V}
  \\
  U_A&=24\me{V}\;.
 }

 Innen a keresett áramok:
 \al{
  I_1 &= \frac{U_A-20\me{V}}{4\me{\Omega}}=1\me{A}
  \\
  I_2 &= \frac{U_A-32\me{V}}{2\me{\Omega}}=-4\me{A}
  \\
  I_3 &= \frac{U_B-U_A}{8\me{\Omega}}=-3\me{A}\;.
 }
 
\fi