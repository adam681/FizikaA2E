\feladat{5}{
 $f=50$\,Hz-es áramkörben $R=50\,\Omega$ nagyságú ohmikus ellenállást és ismeretlen önindukciójú tekercset kapcsolunk sorosan. A fázisszög $\varphi=45^\circ$. Mekkora az öninduktivitás, és mekkora kondenzátor beiktatása szünteti meg a fáziskésést?
}{05fel_01fig}{}

\ifdefined\megoldas

 Megoldás: 

 A forgóvektoros leírásból azonnal látszik, hogy a tekercs impedanciáját megkaphatjuk úgy, mint
 \al{
  X_L
   =R\cdot\tg\varphi
   =R\cdot\tg 45^\circ
   =R\;.
 }
 A tekercs impedanciája:
 \al{
  X_L&=\omega L
  \\
  L&=\frac{X_L}{\omega}
    =\frac{R}{2\pi\cdot f}
    =\frac{50\me{\Omega}}{2\pi\cdot 50\me{Hz}}
    =\frac{1}{2\pi}\me{H}\;.
 }

 A kondenzátor impedanciájának meg kell egyeznie az induktivitás impedanciájával, hogy ne legyen fáziskésleltetés, vagyis
 \al{
  X_C&=X_L
  \\
  \frac{1}{\omega C}&=\omega L
  \\
  C
   &=\frac{1}{\omega^2 L}
    =\frac{1}{\omega R}
    =\frac{1}{2\pi\cdot 50\me{Hz}\cdot 50\me{\Omega}}
    =6,37\cdot 10^{-5}\me{F}\;.
 }
 
\fi