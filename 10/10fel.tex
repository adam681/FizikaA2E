\feladat{10}{
 Egy proton $4\cdot10^6\,\textrm{m/s}$ sebességgel halad át egy $1,7\,\textrm{T}$ nagyságú mágneses téren. A mágneses térrel való kölcsönhatás miatt a protonra $8,2\cdot10^{-13}\,\textrm{N}$ nagyságú erő hat. Mekkora szöget zár be a proton sebességének iránya a mágneses térrel? A proton töltése $e=1,6\cdot10^{-19}$\,C.
}{10fel_01fig}{}

\ifdefined\megoldas

 Megoldás: 

 A megoldáshoz használjuk az előző feladatban szereplő összefüggést:
 \al{
  \sin\varphi
   &=\frac{F_\text{L}}{evB}
    =\frac{8,2\cdot10^{-13}\me{F}}{1,6\cdot 10^{-19}\me{C}\cdot 4\cdot10^6\me{\frac{m}{s}}\cdot 1,7\me{B}}
    =0.754
  \\
   \varphi &=48,9^\circ\;.
 }

 Ez azonban nem egy egyértelmű iránynak felel meg, hiszen a sebességvektor egy kúpfelületen lehet. 
 
\fi