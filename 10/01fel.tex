\feladat{1}{
 \textit{Niels Bohr} 1913-ban felállított modellje szerint a hidrogénatomban a kö\-zép\-pont\-ban lévő proton körül egy elektron kering, attól $R=5,3\cdot10^{-11}\, \textrm{m}$ távolságban, $v=2,2\cdot10^{6}\, \textrm{m/s}$ sebességgel. Határozzuk meg az elektron által keltett mágneses tér nagyságát a proton helyén! Az elektron töltésének nagysága $e=1,6\cdot10^{-19}$\,C.
}{01fel_01fig}{}

\ifdefined\megoldas
  
 Megoldás: 

 Először határozzuk meg, hogy az elektron mekkora áramnak felel meg:
 \al{
  I=\der{Q}{t}
   =\frac{e}{T}
   =\frac{e v}{2\pi R}\;,
 }
 ahol $T$ a keringés ideje. 

 A köráram elemi $\dd\Iv$ darabja 
 \al{
  \dd \Bv=\frac{\mu_0}{4\pi}\frac{\dd\Iv\times \rv}{\abs{\rv}^3}
 }
 mágneses indukciót hoz létre. Mivel a köráram mentén végig merőleges $\rv$ és az áram iránya, így az összes $\dd\Bv$ járulék egy irányba mutat: a síkra merőlegesen felfele. A teljes $\Bv$ vektor így
 \al{
  \Bv
   &=\intl{\text{köráram}}{}\dd \Bv
    =\frac{\mu_0}{4\pi}\intl{\text{köráram}}{}\frac{\dd\Iv\times \rv}{\abs{\rv}^3}
    =\frac{\mu_0 I}{4\pi}\,\ev_z\,\intl{0}{2\pi R}\frac{\dd l}{R^2}
    =\frac{\mu_0 I}{2 R}\,\ev_z
    =\frac{\mu_0 e v}{4\pi R^2}\,\ev_z
  \\
   \abs{\Bv}&=\frac{4\pi\cdot 10^{-7}\me{\frac{V\,s}{A\,m}}\cdot 1,6\cdot10^{-19}\me{C}\cdot 2,2\cdot10^{6}\me{\frac{m}{s}}}{4\pi (5,3\cdot10^{-11}\me{m})^2}
    =157,5\me{T}
   \;.
 }

\fi