\feladat{6}{
 Két hosszú vezetődarabot, melyek tömege méterenként $\mu=40\,\textrm{g}$, szorosan egymás mellé, vízszintesen a mennyezetre függesztünk $l=6\, \textrm{cm}$ hosszú cérnadarabokkal. Mindkét kábelbe $I$ áramot vezetünk, melyek hatására a vezetők egymástól eltávolodnak. Ekkor a két kábelt tartó cérnaszálak $\vartheta = 16^{\circ}$-os szöget zárnak be egymással.
\begin{enumerate}[label=\alph*),itemsep=0pt]
\item A két vezetőben azonos vagy ellenkező irányban folyik az áram?
\item Mekkora az $I$ áramerősség?
\end{enumerate}
}{}{}

\ifdefined\megoldas

 Megoldás: 
 
 \marginfigure{06fel_01fig}
 
 \begin{enumerate}[label=\alph*),itemsep=0pt]
  \item
   A két eset felrajzolásával meggyőződhetünk arról, hogy a vezetékekben ellentétes irányban kell folynia az áramnak. Ha az áram befelé folyik, akkor a bal oldali vezeték által létrehozott mágneses tér lefelé mutat a jobb oldali vezeték helyén. Itt a Lorentz-erő akkor mutat jobbra, vagyis olyan irányba, hogy a vezetéket eltaszítsa a másiktól, ha abban kifelé folyik az áram. A fordított eset is ellenőrizhető, a bal oldali vezetékre is kifele mutató áramot kapunk.
  \item
   Vegyünk $s$ hosszúságú vezetékdarabokat. Ha felírjuk az erők egyenlőségét vízszintes és függőleges irányban:
   \al{
    x: && 0&=F_\text{L}-K\sin\frac{\vartheta}{2}\\
    y: && 0&=K\cos\frac{\vartheta}{2}-\mu\cdot s\cdot g\;,
   }
   ahonnan $K$-t eliminálva
   \al{
    F_\text{L}=\mu s g \cdot\tg\frac{\vartheta}{2}\;.
   }
   
   A Lorentz-erő: $F_\text{L}=B\cdot I\cdot s$, ahol $B$-t az előző feladatok alapján már tudjuk $B=\frac{\mu_0 I}{2\pi d}$. $d$ könnyen meghatározható, hiszen $d=2l\sin\frac{\vartheta}{2}$, amellyel:
   \al{
    F_\text{L}
     =\frac{\mu_0 I^2 s}{4\pi l\sin\frac{\vartheta}{2}}\;.
   } 
   A Lorentz-erőre felírt két kifejezés egyenlőségéből:
   \al{
    \mu s g \cdot\tg\frac{\vartheta}{2} &= \frac{\mu_0 I^2 s}{4\pi l\sin\frac{\vartheta}{2}}
    \\
    I&=\sqrt{\frac{4\pi l \mu g \cdot\tg\frac{\vartheta}{2}\sin\frac{\vartheta}{2}}{\mu_0}}\;.
   }
 \end{enumerate}

 \fi