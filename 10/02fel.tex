\feladat{2}{
 Négyzet alakú, $a=0,4\,\textrm{m}$ élhosszúságú vezető keretben $I=10\, \textrm{A}$ erősségű áram folyik. Számítsuk ki a mágneses tér nagyságát és irányát a keret középpontjában!
}{}{}

\ifdefined\megoldas
  
 Megoldás: 

 A keretet fel tudjuk osztani annak négy oldalára, és az eredő indukciót meg tudjuk adni úgy, mint egy oldal indukciójának a négyszerese. Ezután már csak azt kell meghatároznunk, hogy egy $a$ hosszúságú rúd közepétől $d=\frac{a}{2}$ távolságban mekkora az indukció (lásd 2-B.~ábra).

 Ennek meghatározásához pedig tekintsük a vezetőszakasz $\dd z$ hosszú darabját. Az $O$ pontban létrejövő mágneses indukció merőleges a vezetőt és az $O$ pontot tartalmazó síkra és befelé mutat. Az összes ilyen darabra a mágneses indukció azonos irányba mutat, így az eredő $\Bv$ kiszámolásához elég a vektorok nagyságát összegezni:
 \marginfigure{02fel_01fig}
 \al{
  \dd B
   =\frac{\mu_0}{4\pi}\frac{\abs{\dd \Iv\times \rv}}{\abs{\rv}^3}
   =\frac{\mu_0 I}{4\pi}\frac{\dd z\cdot d}{\left(d^2+z^2\right)^\frac{3}{2}}\;,
 }
 melyet sajnos nem lehet túl praktikusan összegezni a $z\in\left[-\frac{a}{2},\frac{a}{2}\right]$ tartományon úgy, hogy a $z$-t használjuk, mint futó változót. Érdemes áttérni szög szerinti paraméterezésre. Ekkor a $\varphi$ szög a $[-\alpha_\text{h},\alpha_\text{h}]$ tartományt járja be, ahol
 \al{
  \sin\alpha_\text{h}
   =\frac{\frac{a}{2}}{\sqrt{d^2+\frac{a^2}{4}}}\;.
 }
 
 \marginfigure{02fel_02fig}

 A $\dd\varphi$ és a $\dd z$ közötti összefüggést az ábráról olvashatjuk le. A $\varphi$ szögnél lévő $\dd\varphi$ szög alatt látszódó $\dd z$ darab $r=\frac{a}{2\sin\varphi}$ távolságra van az $O$ ponttól, vagyis a kiemelt háromszög befogója $b=r\dd\varphi=\frac{a}{2\sin\varphi}\dd\varphi$. A derékszögű háromszögben $b=\dd z\cos\varphi$, vagyis
 \al{
  \dd z
   =\frac{a}{2\sin\varphi\cos\varphi}\dd\varphi
   =\frac{\frac{a}{2}}{\tg\varphi\cos^2\varphi}\dd\varphi
   =\frac{d}{\cos^2\varphi}\,\dd\varphi\;.
 }
 Így a $\dd B$
 \al{
  \dd B
   &=\frac{\mu_0 I}{4\pi}\frac{\dd \varphi\cdot d}{\left(d^2+z^2\right)^\frac{3}{2}}\cdot \frac{d}{\cos^2\varphi}
    =\frac{\mu_0 I}{4\pi}\frac{\dd \varphi\cdot d}{\left(\frac{d}{\cos\varphi}\right)^3}\cdot \frac{d}{\cos^2\varphi}
    =\frac{\mu_0 I}{4\pi d}\cos\varphi\,\dd \varphi\;. 
 }
 Melyet integrálva:
 \al{
  B
   &=\intl{\text{vezető}}{}\dd B
    =\frac{\mu_0 I}{4\pi d}\intl{-\alpha_\text{h}}{\alpha_\text{h}}\cos\varphi\,\dd \varphi
    =\frac{\mu_0 I}{4\pi d}\left[\sin\varphi\right]_{-\alpha_\text{h}}^{\alpha_\text{h}}
    =\frac{\mu_0 I}{2\pi d}\sin\alpha_\text{h}
  \\
   &=\frac{\mu_0 I}{2\pi d}\frac{\frac{a}{2}}{\sqrt{d^2+\frac{a^2}{4}}}\;,
 }
 és mivel itt $d=\frac{a}{2}$, így
 \al{
  B=\frac{\mu_0 I}{2\pi a}\sqrt{2}\;.
 }

 A teljes hurokra ennek a négyszeresét kapjuk, így
 \al{
  B_\text{hurok}
   =\frac{2\mu_0 I}{\pi a}\sqrt{2}\;.
 }

\fi