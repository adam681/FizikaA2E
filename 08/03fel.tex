\feladat{3}{
 Egy $V_{0}$ elektromotoros erejű, $R_\text{b}$ belső ellenállású telepre egy $R$ nagyságú ellenállást kötünk. 
 \begin{enumerate}[label=\alph*),itemsep=0pt]
 \item Mekkora terhelő ellenállás esetén lesz maximális a telepből kivett teljesítmény?
 \item Mikor maximális a hatásfok?
 \end{enumerate}
}{}{}

\ifdefined\megoldas
 
 Megoldás: 

 \begin{enumerate}[label=\alph*),itemsep=0pt]
  \item
   A hasznos teljesítmény az, ami a telepre kapcsolt $R$ ellenálláson esik: $P_\text{hasznos}=V_R\cdot I$. Ehhez először ki kell számolni az $R$ ellenálláson eső feszültséget, és az áramkörben folyó áramot. 

   A telepet terhelő eredő ellenállás: $R_\text{e}=R+R_\text{b}$, vagyis az áram:
   \al{
    I=\frac{V}{R_\text{e}}
    =\frac{V}{R+R_\text{b}}\;.
   }

   Mivel ez egy soros feszültségosztó, így az ellenálláson eső feszültség:
   \al{
    V_R=\frac{R}{R+R_\text{b}} V\;.
   }
   Ezek alapján a hasznos teljesítmény:
   \al{
    P_\text{hasznos}
     = \frac{R}{(R+R_\text{b})^2} V^2\;.
   }

   Ennek a maximumát keressük az $R$ változtatása mellet. Ehhez deriváljuk a kifejezést $R$ szerint:
   \al{
    \der{P_\text{hasznos}}{R}
     &=\der{}{R}\left(\frac{R}{(R+R_\text{b})^2} V^2\;.\right)
      =V^2\cdot\left(\frac{1}{(R+R_\text{b})^2}-2\frac{R}{(R+R_\text{b})^3}\right)\;.
   }
   A szélsőérték meglétének feltétele, hogy az első derivált nulla legyen:
   \al{
    0&=V^2\cdot\left(\frac{1}{(R+R_\text{b})^2}-2\frac{R}{(R+R_\text{b})^3}\right)\\
    0&=1-2\frac{R}{R+R_\text{b}}\\
    R&=R_\text{b}\;.
   }
   Ahhoz, hogy ez egy maximum legyen, elégséges azt megvizsgálni, hogy a második derivált negatív-e ezen a helyen:
   \al{
    \der{^2 P_\text{hasznos}}{R^2}
     &=V^2\cdot\der{}{R}\left(\frac{1}{(R+R_\text{b})^2}-2\frac{R}{(R+R_\text{b})^3}\right)\\
     &=V^2\cdot\left(-2\frac{1}{(R+R_\text{b})^3}-2\frac{1}{(R+R_\text{b})^3}+6\frac{R}{(R+R_\text{b})^4}\right)\;,
    \\
    \left.\der{^2 P_\text{hasznos}}{R^2}\right|_{R=R_\text{b}}
     &=V^2\cdot\left(-4\frac{1}{8 R_\text{b}^3}+6\frac{R_\text{b}}{16 R_\text{b}^4}\right)
      =-V^2\cdot \frac{1}{8 R_\text{b}^3} < 0\;.
   }

   Valóban negatív, vagyis a teljesítmény tényleg maximális.

  \item
   A hatásfok a hasznos teljesítmény és a teljes befektetett teljesítmény aránya. A telep által leadott teljesítmény itt $P_\text{telep}=VI$, vagyis 
   \al{
    \eta
     =\frac{P_\text{hasznos}}{P_\text{telep}}
     =\frac{\frac{R}{(R+R_\text{b})^2} V^2}{\frac{1}{R+R_\text{b}} V^2}
     =\frac{R}{R+R_\text{b}}\;,
   }
   melynek maximumát hasonlóan kereshetjük meg:
   \al{ 
    \der{\eta}{R}
     &=\frac{1}{R+R_\text{b}}-\frac{R}{(R+R_\text{b})^2}
      =0
    \\
    1&=\frac{R}{R+R_\text{b}}
    \\
    R_\text{b}&=0\;.
   }
   \marginfigure{03fel_01fig}
   
   Láthatjuk, hogy nincs olyan $R$ érték, amelyre az egyenletet teljesíteni tudjuk ($R$ kiesik), vagyis az $\eta(R)$ függvénynek nincs szélsőértéke. Ha ábrázoljuk a görbét, könnyen láthatjuk, hogy az fizikai ($R>0$) tartományban a hatásfok $R$ növelésével egyre nő, és minél nagyobb a terhelő ellenállás, annál nagyobb lesz a hatásfok is, de az $\eta=1$-et csak aszimptotikusan tudjuk elérni. 
 \end{enumerate}

\fi