\feladat{5}{
 Két ellenállás közül az egyik $R_1=40$\,k$\Omega$ nagyságú és $P_1=4\,\textrm{W}$ teljesítményű, a másik $R_2=10$\,k$\Omega$ nagyságú és szintén $P_2=4\,\textrm{W}$ teljesítményű. Mekkora feszültséget kapcsolhatunk maximálisan az ellenállásokra, ha sorba kötjük őket?
}{}{}

\ifdefined\megoldas

 Megoldás: 

 Ha sorba kapcsoljuk őket, akkor ugyanakkora áram folyik át rajtuk. Számoljuk ki az egyik és a másik esetben is ezt a kritikus értéket, és a kettő közül a kisebbik lehet a maximális rákapcsolt áram. 

 A teljesítmény: $P=VI=I^2 R$ felhasználva az $R=V/I$ Ohm-törvényt. A megadott teljesítmények névleges teljesítmények: ezek azt jelentik, hogy legfeljebb mekkora teljesítményt vesznek fel a hálózatból a fogyasztók. Innen a legnagyobb átfolyatható áramok:
 \al{
  I_{1,\text{max}}
   &=\sqrt{\frac{P_1}{R_1}}
    =\sqrt{\frac{4\me{W}}{40\me{k\Omega}}}
    =0,01\me{A}\;,
  \\
  I_{2,\text{max}}
   &=\sqrt{\frac{P_2}{R_2}}
    =\sqrt{\frac{4\me{W}}{10\me{k\Omega}}}
    =0,02\me{A}\;,
 }
 tehát a teljes áramkörön $I_\text{max}=0,01\me{A}$ folyhat át. Ekkor a teljes áramkörre kapcsolt feszültség:
 \al{
  V=I_\text{max} R_1+I_\text{max} R_2
   =0,01\me{A}\cdot 50\me{k\Omega}
   =500\me{V}\;.
 }
 
\fi