\feladat{10}{
 Az ábrán látható kapcsolásban mekkorának válasszuk az $R_{x}$ ellenállást, hogy a $K$ kapcsoló zárása ne befolyásolja az $I$ áram értékét?
}{10fel_01fig}{1}

\ifdefined\megoldas

 Megoldás: 

 A kapcsoló bekapcsolása akkor nem módosítja a folyó áramot, ha a kapcsoló két vége között bekapcsolt állapotban sem folyik áram, ami akkor lehetséges, ha a két végpont ugyanakkora potenciálú.

 Azt kiszámoltuk korábban, hogy a soros ágakban az ellenállások arányában esik a feszültség, vagyis 
 \al{
  V_1&=\frac{R_1}{R_1+R_2}V
  &
  V_x&=\frac{R_x}{R_x+R_3}V\;.
 }

 Akkor lesz ekvipotenciális a kapcsoló két végpontja, ha az $R_1$ és az $R_x$ ellenállásokon ugyanakkora feszültség esik, vagyis ha
 \al{
  V_1&=V_x
  \\
  \frac{R_1}{R_1+R_2}V&=\frac{R_x}{R_x+R_3}V
  \\
  R_1\cdot (R_x+R_3)&=R_x\cdot(R_1+R_2)
  \\
  R_1 R_3=R_x R_2
  \\
  \frac{R_1}{R_2}&=\frac{R_x}{R_3}\;.
 }
 Tehát akkor nem folyik áram a kapcsolón, ha a soros ágakban az ellenállások aránya megegyezik.
 
\fi