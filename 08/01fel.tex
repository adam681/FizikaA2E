\feladat{1}{
 Az ábrán látható áramkörben határozzuk meg az $I$ áramerősséget!
}{01fel_01fig}{1}

\ifdefined\megoldas
 
 Megoldás: 

 Először ki kell számolnunk az öt ellenállásból álló hálózat eredő ellenállását. Vegyük észre, hogy a 3-as és az 5-ös ellenállás végei rövidre vannak zárva, így azokon sosem folyik áram, vagyis azok nem járulnak hozzá az eredő ellenálláshoz. 

 Így összesem három ellenállás marad, az 1-es, a 2-es és a 4-es, melyek közül a 2-es és a 4-es párhuzamosan vannak kapcsolva:
 \al{
  R_{2,4}
   =\frac{1}{\frac{1}{R_2}+\frac{1}{R_4}}
   =\frac{1}{2\cdot\frac{1}{R}}
   =\frac{1}{2}R\;,
 }
 illetve ezekkel sorosan van kapcsolva az 1-es:
 \al{
  R_\text{e}
   =R+\frac{1}{2}R
   =\frac{3}{2}R\;,
 }

 Innen a telepen átfolyó áram:
 \al{
  I=\frac{U}{R_\text{e}}
   =\frac{U}{\frac{3}{2}R}
   =\frac{2}{3}\frac{U}{R}\;.
 }

\fi