\feladat{7}{
 Egy autóakkumulátort töltés céljából $13\,\textrm{V}$ elektromotoros erejű, $0,09\,\Omega$ belső ellenállású töltőre kapcsolunk. Az akkumulátor belső ellenállása $0,01\,\Omega$, elektromotoros ereje $12\,\textrm{V}$.
 \begin{enumerate}[label=\alph*),itemsep=0pt]
  \item Mekkora a töltőáram?
  \item Mekkora a töltő által leadott teljesítmény?
  \item Mennyi az akkumulátor és a töltő melegedésére fordítódó teljesítmény?
  \item Mennyi az akkumulátor töltésére fordítódó teljesítmény?
 \end{enumerate}
}{07fel_01fig}{}

\ifdefined\megoldas

 Megoldás: 

 \begin{enumerate}[label=\alph*),itemsep=0pt]
  \item 
   Az akkumulátort úgy dugjuk a töltőre, hogy annak negatív pólusát a töltő negatív pólusához, a pozitívat pedig a pozitívhoz kapcsoljuk. Az áramkör egy hurokból áll. A huroktörvény:
   \al{
    0&=-U_\text{töltő}+IR_\text{b,töltő}+IR_\text{b,akku}+U_\text{akku}
    \\
    I&=\frac{U_\text{töltő}-U_\text{akku}}{R_\text{b,töltő}+R_\text{b,akku}}
      =\frac{13\me{V}-12\me{V}}{0,09\me{\Omega}+0,01\me{\Omega}}
      =10\me{A}\;.
   }
   
  \item
   A töltő által leadott teljesítmény:
   \al{
    P_\text{töltő}
     =U_\text{töltő}I
     =13\me{V}\cdot 10\me{A}
     =130\me{W}\;.
   }
  
  \item
   Az akkumulátor és a töltő melegedését a belső ellenállásokon termelődő Joule-hő okozza. Ez veszteségként jelenik meg:
   \al{
    P_\text{veszteség}
     &=I^2\cdot R_\text{b,akku}+I^2\cdot R_\text{b,töltő}
    \\
     &=(10\me{A})^2\cdot 0,01\me{\Omega)}+(10\me{A})^2\cdot 0,09\me{\Omega)}
      =10\me{W}\;.
   }
   
  \item 
   Az akkumulátor töltésére 
   \al{
    P_\text{töltés}
     =U_\text{akku}I
     =12\me{V}\cdot 10\me{A}
     =120\me{W}
   }
   teljesítmény fordítódik.
   
   Vegyük észre, hogy töltő által leadott teljesítmény megegyezik a hasznos teljesítmény és a veszteség összegével: ez egy példa energiamegmaradás törvényére.
 \end{enumerate}

\fi