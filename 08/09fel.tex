\feladat{9}{
 Mekkora az ábrán feltüntetett $A$ és $B$ pontok között mérhető feszültség, ha a telep belső ellenállása elhanyagolható?
}{09fel_01fig}{1}

\ifdefined\megoldas

 Megoldás: 

 Ahhoz, hogy meghatározzuk az $A$ és a $B$ pontok közötti feszültséget, az áramkör mentén be kell járnunk egy utat a kép pont között, és meg kell néznünk, hogy mekkora feszültség esik az egyes áramköri elemeken, amelyeken áthaladunk. Válasszuk az $A\to C\to B$ útvonalat.

 Az állandósult állapotban a $C_1$ és $C_2$ kondenzátorok fel vannak töltődve, azokon áram nem folyik. Így áram csak az ellenállásokat tartalmazó ágban van:
 \al{
  I=\frac{V}{R_1+R_2}\;.
 }
 Az $R_1$ ellenálláson eső feszültség:
 \al{
  V_1=\frac{R_1}{R_1+R_2}V\;,
 }
 melyet pozitív előjellel kell figyelembe venni, hiszen az áram arra folyik, amerre haladunk ($A\to C$).

 Mivel a kondenzátorok sorba vannak kapcsolva, így a töltésük megegyezik: $Q_1=Q_2=Q$. Eredő kapacitásuk
 \al{
  C=\frac{1}{\frac{1}{C_1}+\frac{1}{C_2}}\;,
 }
 mellyel a töltés
 \al{
  Q=CV
   =\frac{V}{\frac{1}{C_1}+\frac{1}{C_2}}\;,
 }
 vagyis az 1-es kondenzátoron eső feszültség
 \al{
  V_{C,1}
   =\frac{Q}{C_1}
   =\frac{V}{\frac{1}{C_1}+\frac{1}{C_2}}\frac{1}{C_1}
   =\frac{C_2}{C_1+C_2}V\;.
 }
 Ezt viszont negatív előjellel kell figyelembe venni, hiszen a feszültség növekszik,  ahogy a kondenzátoron átlépünk. Tehát a $B$ pont $A$ ponthoz viszonyított feszültsége :
 \al{
  V_\text{AB}
   =\left(\frac{R_1}{R_1+R_2}-\frac{C_2}{C_1+C_2}\right)V\;.
 }

\fi