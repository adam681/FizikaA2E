\feladat{2}{
 Az ábrán látható áramkörben számítsuk ki az egyes ágakban folyó áram\-erős\-sé\-ge\-ket! 
}{02fel_01fig}{1}

\ifdefined\megoldas
 
 Megoldás: 

 A megoldáshoz a Kirchhoff törvényeket fogjuk használni. Az áramkör két csomópontot tartalmaz, írjuk fel ezekre a csomóponti törvényt:
 \al{
  A: && 0&=I_1-I_2-I_3     \label{eq:2-cs1}\\
  B: && 0&=-I_1+I_2+I_3 \;.\label{eq:2-cs2}
 }

 Ezután írjuk fel az áramkörben Kirchhoff II. törvényét. Összesen három hurokra lehet ezt megtenni:
 \al{
  V_1\to R_1\to R_3:        && 0&=-V_1+I_1 R_1+I_3 R_3       \label{eq:2-h1}\\
  R_3\to R_2\to V_2:        && 0&=-I_3 R_3+I_2 R_2+V_2       \label{eq:2-h2}\\
  V_1\to R_1\to R_2\to V_2: && 0&=-V_1+I_1 R_1+I_2 R_2+V_2   \label{eq:2-h3}
 }

 A Kirchhoff-törvények öt egyenletet adnak, melyek három ismeretlent tartalmaznak ($I_1$, $I_2$ és $I_3$). Ezeknek csak akkor lehet megoldása, ha az egyenletek összefüggőek. Azt azonnal láthatjuk, hogy a csomóponti egyenlet összefüggő, az egyik a másik $-1$-szerese, illetve az is észrevehető, hogy az első két hurokegyenlet összege a harmadikat adja. Tehát ha \eqaref{eq:2-cs1}, \eqaref{eq:2-h1} és \eqaref{eq:2-cs2} egyenletek nem ellentmondóak, akkor egyértelmű megoldást adnak a három ismeretlenre.

 Az egyenletrendszer:
 \al{
  0&=I_1-I_2-I_3\\
  0&=-32\me{V}+I_1 \cdot 2\me{\Omega}+I_3 \cdot 8\me{\Omega}\\
  0&=-I_3 \cdot 8\me{\Omega}+I_2 \cdot 4\me{\Omega}+20\me{V}\label{eq:2-1}
 }

 Az első egyenletből $I_1=I_2+I_3$, amit a másodikba behelyettesítve:
 \al{
  0&=-32\me{V}+I_2\cdot 2\me{\Omega}+I_3 \cdot 10\me{\Omega}\label{eq:2-2}\;.
 }
 \Eqaref{eq:2-2}-es egyenlet kettővel megszorozva, majd abból \eqaref{eq:2-1}-et kivonva:
 \al{
  0&=-64\me{V}+I_3 \cdot 20\me{\Omega}-\big(-I_3 \cdot 8\me{\Omega}+20\me{V}\big)\\
  0&=-84\me{V}+I_3 \cdot 28\me{\Omega}\\
  I_3&=3\me{A}\;.
 }
 Ez visszahelyettesítve \eqaref{eq:2-2}-be:
 \al{
  0&=-32\me{V}+I_2\cdot 2\me{\Omega}+3\me{A} \cdot 10\me{\Omega}\\
  I_2&=1\me{A}\;.
 }
 \Eqaref{eq:2-cs1} alapján:
 \al{
  I_1&=4\me{A}\;.
 }

 Tehát az ellenállásokon az ábrán szereplő nyilak irányában folyik át rendre $4\me{A}$, $1\me{A}$ és $3\me{A}$.

\fi