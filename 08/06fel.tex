\feladat{6}{
 Az $V_\text{mh}=5\,\textrm{V}$ méréshatárú, $R_\text{b}=800\,\Omega$ belső ellenállású feszültségmérővel sorba kapcsolunk egy $R_\text{e}=12,5$\,k$\Omega$ nagyságú ellenállást. Mekkorára növekszik így a műszer méréshatára?
}{}{}

\ifdefined\megoldas

 Megoldás: 

 A műszer az ő két sarka közötti feszültséget méri. Ha a műszerrel sorba kapcsolunk egy másik ellenállást, akkor nem a teljes feszültség fog a műszeren esni. Soros feszültségosztás esetében:
 \al{
  V_\text{mért}
   &=\frac{R_\text{b}}{R_\text{e}+R_\text{b}} V_\text{ráadott}\;.
 }
 Ha a műszer méréshatára $5\me{V}$, akkor a $V_\text{mért}$ értéke eshet $0$--$5\me{V}$ közé. Ekkor a $V_\text{ráadott}$ maximális értéke:
 \al{
  V_{\text{max},\text{ráadott}}
   =\frac{R_\text{e}+R_\text{b}}{R_\text{b}}V_\text{mh}
   =\frac{12,5\me{k\Omega}+800\me{\Omega}}{800\me{\Omega}}5\me{V}
   =83,125\me{V}\;.
 }

\fi