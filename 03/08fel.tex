\feladat{8}{
 Tegyük fel, hogy a térben a térfogati töltéssűrűség gömbszimmetrikus, tehát az csak az origótól mért távolságtól függ, vagyis $\varrho(x,y,z)=f\big(\sqrt{x^2+y^2+z^2}\big)$ alakú. Hogyan fejezhető ki az elektromos térerősség $f$ se\-gít\-sé\-gé\-vel?
}{}{}

\ifdefined\megoldas
   

 Megoldás: 

 Itt a töltéssűrűség csak az $r=\sqrt{x^2+y^2+z^2}$-től, vagyis az origótól mért távolságtól függ. Így a rendszer 
 \begin{enumerate}[itemsep=0pt]
  \item tetszőleges origót átmenő tengely körül tetszőleges szöggel elforgatható,
  \item tetszőleges az origót tartalmazó síkra tükrözhető
 \end{enumerate}
 úgy, hogy a rendszer a transzformáció után önmagába megy át, vagyis a rendszer gömbszimmetrikus. Gondolkodjunk gömbkoordináta-rendszerben, a megfelelő koordináták: $r$, $\vartheta$ és $\varphi$. Az 1. szimmetria következménye, hogy semmilyen mennyiség sem függhet a $\vartheta$ és a $\varphi$ szögektől, a 2. szimmetriáé pedig, hogy a térerősség csak sugárirányba mutathat, azaz
 \al{
  \Ev(r,\vartheta,\varphi)
   &=\Ev(r)
    =E(r)\ev_\text{r}
    \;.
 }

 A megfelelő Gauss-felület egy origó középpontú, $r$ sugarú gömb. A Gauss-tételben szereplő felületi integrál:
 \al{
  \iint\limits_{\text{gömb}}\Ev(\rv)\;\dd^2\fv
   &
   \overset{(1)}{=}\iint\limits_{\text{gömb}}E(r)\;\dd^2 f
   \overset{(2)}{=}E(r)\iint\limits_{\text{gömb}}\;\dd^2 f
    =E(r)\cdot 4r^2\pi\;,
 }
 hiszen (1) a gömb felületén számolt felületi integrálnál a felületvektor párhuzamos a térerősséggel minden pontban, illetve mivel a térerősség csak az $r$ koordinátától függ, így a felületen a térerősség nagysága mindenhol ugyanakkora (2), vagyis az kiemelhető az integrálás alól, a felületelemek összege pedig az alaplap felületét adja.

 A körbezárt töltés:
 \al{
  \iiint\limits_\text{gömb}\varrho(\rv)\,\dd^3\rv
   &\overset{(1)}{=}
    \intl{0}{2\pi}
    \intl{0}{\pi}
    \intl{0}{r}  \varrho(r')r'^2\,\sin\vartheta\dd r'\dd\vartheta\dd \varphi
   \\
   &\overset{(2)}{=}
    \intl{0}{\pi}\sin\vartheta\dd \vartheta
    \intl{0}{2\pi} \dd\varphi
    \intl{0}{r}\varrho(r')r'^2\,\dd r'
    \overset{(3)}{=}
    4\pi\intl{0}{r}r'^2\varrho(r')\,\dd r'\;,
 }
 ahol $r$ az integrálás határa és $r'$ az integrálási változó.

 A Gauss-tétel alapján:
 \al{
  E(r)
   =\frac{1}{\ep_0}\frac{1}{r^2}\intl{0}{r}r'^2\varrho(r')\,\dd r'\;.
 }

 \fi