\feladat{5}{
 Milyen erőteret hogy létre két egymásra merőleges végtelen sík, ha rajtuk egyenletesen elosztva $\sigma$ és $2\sigma$ felületi töltéssűrűség van. 
}{}{}

\ifdefined\megoldas
   

 Megoldás: 

 \marginfigure{05fel_01fig.tikz}
  
 Először határozzuk meg egy $\sigma$ felületi töltéssűrűséggel rendelkező síklap térerősségét. Ehhez is a Gauss-törvényt fogjuk használni. Szimmetriamegfontolások alapján megállíthatjuk, hogy a térerősség csak a síktól való $z$ távolságtól függhet, illetve hogy a térerősségnek erőlegesnek kell lennie a síkra. 

 A Gauss-törvényt egy henger alakú felületre fogjuk felírni. Legyen ennek alapterülete $A$, magassága $2z$, és helyezzük el úgy, hogy a töltött sík ezt pont felezze. A befoglalt töltés $Q_\text{in}=\sigma A$. A térerősségnek felületi integrálja csak a két alaplapon fog járulékot adni, hiszen az a palásttal párhuzamos. Mind a két alaplapon a felületvektor és a térerősség egy irányba mutat, így:
 \al{
  \frac{1}{\varepsilon_0}\sigma A
   &=\iint\limits_{\text{henger}}\Ev(\rv)\;\dd^2\fv
    =2\iint\limits_{\text{alaplap}}\Ev(z)\;\dd^2\fv
    =2E(z)A
  \\
   E(z)
    &=\frac{\sigma}{2\ep_0}\;.
 }

 Tehát láthatjuk, hogy a síktól való távolságtól függetlenül a térerősség állandó nagyságú. 

 Ha két ilyen lemezt merőlegesen állítunk egymásra, akkor az egyes lemezek által létrehozott térerősségeket vektoriálisan össze kell adni. A nagysága mindenhol ugyanakkora:
 \al{
  E
   =\sqrt{E_1^2+E_2^2}
   =\sqrt{\left(\frac{\sigma}{\ep_0}\right)^2+\left(\frac{\sigma}{2\ep_0}\right)^2}
   =\frac{\sqrt{5}\sigma}{2\ep_0}\;.
 }
 A térerősségvektor:
 \al{
  \Ev(x,y)
   =\begin{cases}
     \frac{\sigma}{2\ep_0}\big(\phantom{-}1,\phantom{-}2,0\big) & 0<x\;,\;0<y \\
     \frac{\sigma}{2\ep_0}\big(\!\!-\! 1,\phantom{-}2,0\big) & 0>x\;,\;0<y \\
     \frac{\sigma}{2\ep_0}\big(\phantom{-}1,\!\!-\!2,0\big) & 0<x\;,\;0>y \\
     \frac{\sigma}{2\ep_0}\big(\!\!-\! 1,\!\!-\!2,0\big) & 0>x\;,\;0>y 
    \end{cases}
 }
 
\fi