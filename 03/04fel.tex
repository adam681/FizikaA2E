\feladat{4}{
 Egymástól $d$ távolságban párhuzamosan elhelyezett két igen hosszú fonalat egyenletesen töltök fel $+\lambda$ és $-\lambda$ lineáris töltéssűrűséggel. Határozzuk meg a térerősséget abban a pontban, amely a két fonalat magában foglaló síktól $x$ távolságban fekszik a szimmetriasíkban. 
}{04fel_01fig}{}

\ifdefined\megoldas
  
 
Megoldás: 

Az előző feladat megoldása alapján tudjuk, hogy a $\lambda$ töltéssűrűséggel rendelkező vonaltöltés elektromos térerőssége:
\al{
 \Ev(r)=\frac{\lambda}{2\pi\ep_0}\frac{1}{r}\;\ev_\text{r}\;.
}

A két vonaltöltés által létrehozott térerősséget a megfelelő módon vektoriálisan kell összegeznünk. Azt láthatjuk, hogy a térerősség $x$ komponensei kiejtik egymást, így csak a $-y$ irányba mutató vetületeket kell összeadnunk. 
\al{
 E(x)
  &=2E\Bigg(\sqrt{x^2+\left(\frac{d}{2}\right)^2}\Bigg)\cdot\cos\alpha
   =\frac{\lambda}{\pi\ep_0}\frac{1}{\sqrt{x^2+\left(\frac{d}{2}\right)^2}}\cdot\cos\alpha
 \\
  &=\frac{\lambda}{\pi\ep_0}\frac{1}{\sqrt{x^2+\left(\frac{d}{2}\right)^2}}\cdot\frac{\frac{d}{2}}{\sqrt{x^2+\left(\frac{d}{2}\right)^2}}
   =\frac{\lambda}{2\pi\ep_0}\frac{d}{x^2+\left(\frac{d}{2}\right)^2}\;.
}
Vagyis 
\al{
 \Ev(x)
  &=\frac{\lambda}{2\pi\ep_0}\frac{d}{x^2+\left(\frac{d}{2}\right)^2}\cdot(-\ev_\text{y})\;.
}

\fi