\feladat{10}{
 Egy $C_1$ kapacitású, $V$ feszültségre töltött kondenzátor egyik fegyverzetét egy $R$ ellenálláson keresztül egy másik, $C_2$ kapacitású töltetlen kondenzátor egyik fegyverzetére kötjük. A két kondenzátor szabad fegyverzeteit rövidre zárjuk.
 \begin{enumerate}[label=\alph*),itemsep=0pt]
  \item A tranziensek lecsengése után mekkora feszültséget mérhetünk a kondenzátorokon?
  \item Mekkora az állandósult állapotban a kondenzátorok teljesítménye?
  \item Mekkora az állandósult állapotban a kondenzátorok energiája?
 \end{enumerate}
}{10fel_01fig}{}

\ifdefined\megoldas

 Megoldás: 

 \begin{enumerate}[label=\alph*),itemsep=0pt]
  \item 
   Legyen kezdetben $Q_1=C_1 V$ az 1. kondenzátoron a töltés. Az össztöltés megmarad, így az egyensúly beálltával is $Q_1=Q'_1+Q'_2$. Kirchhoff II.\! törvénye értelmében az egyensúlyban $V'_{\text{C},1}=V'_{\text{C},2}=V'$. Innen
  \al{
   C_1 V&=C_1 V'+C_2 V'\\
   V'=\frac{C_1}{C_1+C_2}V\;.
  }
  
  \item
   Mivel áram nem folyik az állandósult állapotban, így a kondenzátorok teljesítménye nulla.

  \item 
   A kondenzátorok energiája:
   \al{
    E_1
     &=\frac{1}{2}C_1V'^2
      =\frac{1}{2}\frac{C_1^3}{(C_1+C_2)^2}V^2
    \\
    E_2
     &=\frac{1}{2}C_2V'^2
      =\frac{1}{2}\frac{C_2 C_1^2}{(C_1+C_2)^2}V^2\;.
   }
  
 \end{enumerate}
 
\fi