\feladat{3}{
 Két $l=15\,\textrm{cm}$ hosszúságú koaxiális henger közötti teret szilícium tölt ki. A belső henger sugara $R_1=0,5\,\textrm{cm}$, a külső hengeré pedig $R_2=1,75\,\textrm{cm}$. Számítsuk ki a hengerpalástok között mérhető ellenállást! A szilícium fajlagos ellenállása $\varrho_\text{Si}=640\,\Omega\textrm{m}$.
}{03fel_01fig}{}

\ifdefined\megoldas
  
 Megoldás: 

 A feladat megoldása során megpróbáljuk visszavezetni a problémát az egyszerű esetre. Az áram folyási irányára merőlegesen felosztjuk a vezetőt: válasszunk egy $r$ sugarú, $\dd r$ vastagságú és $l$ hosszú darabot. Ennek ellenállását egyszerűen tudjuk számolni:
 \al{
  \dd R(r)
   &=\varrho_\text{Si}\frac{\dd r}{A}
    =\varrho_\text{Si}\frac{\dd r}{2r\pi l}\;.
 } 
 Az ilyen darabokon egymás után folyik keresztül az áram, vagyis ezek sorosan vannak kapcsolva. A teljes ellenállás az ilyen ellenállások összege:
 \al{
  R
   &=\intl{R_1}{R_2}\dd R(r)
    =\intl{R_1}{R_2}\varrho_\text{Si}\frac{\dd r}{2r\pi l}
    =\frac{\varrho_\text{Si}}{2\pi l}\intl{R_1}{R_2}\frac{1}{r}\, \dd r
    =\frac{\varrho_\text{Si}}{2\pi l}\big[\ln r\big]_{R_1}^{R_2}
    =\frac{\varrho_\text{Si}}{2\pi l}\ln \frac{R_2}{R_1}
   \;.
 }
 
\fi