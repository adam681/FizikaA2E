\feladat{6}{
 $V$ feszültséget kapcsolunk két sorosan kapcsolt, $R_1$ és $R_2$ nagyságú ellenállásra. Számítsuk ki az egyes ellenállásokon eső feszültségeket! Írjuk fel a feszültségek arányát!
}{06fel_01fig}{}

\ifdefined\megoldas

 Megoldás: 

 A két ellenállás eredő ellenállása:
 \al{
  R_\text{e}
   =R_1+R_2\;,
 }
 vagyis a teljes átfolyó áram:
 \al{
  I=\frac{V}{R_\text{e}}
   =\frac{V}{R_1+R_2}\;.
 }

 Az egyik és a másik ellenálláson eső feszültség:
 \al{
  V_1
   &=I\cdot R_1
    =\frac{R_1}{R_1+R_2}\cdot V
  &
  V_2
   &=I\cdot R_2
    =\frac{R_2}{R_1+R_2}\cdot V\;,
 }
 vagyis a feszültségek aránya:
 \al{
  \frac{V_1}{V_2}=\frac{R_1}{R_2}\;.
 }
 Soros kapcsolás esetében a feszültségek az ellenállások arányában oszlanak meg.

\fi