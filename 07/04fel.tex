\feladat{4}{
 Az amperben mért áramerősséget az idő függvényében az $I=2t^2+3t+7$ összefüggés írja le, ahol az időt másodpercben mérjük. Mekkora nagyságú töltés áramlik át a vezető keresztmetszetén $t_{1}=1\,\textrm{s}$ és $t_{2}=4\,\textrm{s}$ között?
}{04fel_01fig}{}

\ifdefined\megoldas
  
 Megoldás: 

 Az áram a vezető teljes felületén időegység alatt átáramló töltések száma. Az áram integrálja adja meg az átáramló töltés mennyiségét:
 \al{
  Q_\text{tot}
   &=\intl{t_1}{t_2}I(t)\,\dd t
    =\intl{t_1}{t_2}(2t^2+3t+7)\,\dd t
    =\left[2\frac{t^3}{3}+3\frac{t^2}{2}+7t\right]_{t_1}^{t_2}
  \\
   &= \left(2\frac{4^3}{3}+3\frac{4^2}{2}+7\cdot 4\right)
     -\left(2\frac{1^3}{3}+3\frac{1^2}{2}+7\cdot 1\right)
    =85,5\,[\mathrm{C}]\;.
 }
 
\fi