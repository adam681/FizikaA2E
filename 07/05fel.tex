\feladat{5}{
 Két darab $1,5$\,mm$^{2}$ keresztmetszetű vezetéket sorba kapcsolunk. Az első vezeték $5$\,m hosszú és rézből készült, a második pedig $15$\,m hosszú és alumíniumból készült. Határozzuk meg az összekapcsolt vezetékek ellenállását! $V=2$\,V feszültség hatására mekkora áram folyik a vezetékben? ($\varrho_{\textrm{Cu}}$ = $0,018\,\Omega$mm$^{2}$/m, $\varrho_{\textrm{Al}}=0,027\,\Omega$mm$^{2}$/m)
}{}{}

\ifdefined\megoldas

 Megoldás: 

 A két vezetékdarab ellenállása:
 \al{
  R_\text{Cu}
   &=\varrho_\text{Cu}\frac{l_\text{Cu}}{A}
    =0,018\me{\Omega \frac{mm^2}{m}}\frac{5\me{m}}{1,5\me{mm^2}}
    =0,06\me{\Omega}\;,
  \\
  R_\text{Al}
   &=\varrho_\text{Al}\frac{l_\text{Al}}{A}
    =0,027\me{\Omega \frac{mm^2}{m}}\frac{15\me{m}}{1,5\me{mm^2}}
    =0,27\me{\Omega}\;.
 }

 A sorosan kapcsolt vezetékek ellenállása összeadódik, így a teljes vezeték ellenállása:
 \al{
  R=R_\text{Cu}+R_\text{Al}
   =0,33\me{\Omega}\;.
 }
 A vezetéken átfolyó áram:
 \al{
  I=\frac{V}{R}
   =\frac{2\me{V}}{0,33\me{\Omega}}
   =6,06\me{A}\;.
 }

\fi