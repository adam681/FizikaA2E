\feladat{4}{
 Tegyük fel, hogy a térben a térfogati töltéssűrűség csak az $x$--$y$ síktól mért távolságtól függ, tehát $\varrho(x,y,z)=f\big(\abs{z}\big)$ valamilyen $f$ függvényre. Fejezzük ki $f$ segítségével a potenciált a tér minden pontjában!
}{}{}

\ifdefined\megoldas
  
 Megoldás: 

 Használjuk fel a 3.~feladatsor 6.~feladatának eredményét. Ott kiszámoltuk, hogy ebben az esetben a térerősség 
 \al{ 
  \Ev(z) = \frac{1}{\ep_0}\intl{0}{z}\varrho(z')\,\dd z'\cdot \sgn(z)\cdot \ev_z\;.
 }

 A töltéselrendezés szimmetriája miatt a potenciál is szimmetrikus lesz az $x$--$y$ síkra, vagyis $U(z)=U(\abs{z})$. A potenciál a $z>0$ tartományban
 \al{ 
  U(z)
   &= U(0) -\intl{0}{z}\Ev(\rv')\,\dd\rv'\;,
 }
 ahol az integrálás egy olyan úton megy végig, amelyik a $z'=0$ magasságban kezdődik, és $z'=z$ magasságig jut. Sok ilyen integrálási út létezik, azonban a potenciálelmélet alaptétele kimondja, hogy a potenciál értéke független ettől az integrálási úttól, az csak a kezdeti és a végponttól függ. Válasszuk akkor ezt az utat úgy, hogy az integrálás a lehető legegyszerűbb legyen: haladjuk a $z$ tengellyel párhuzamosan. 
 \al{ 
  U(z)
   &= U(0) -\intl{0}{z}\Ev(\rv')\ev_z\dd z'
    = U(0) -\intl{0}{z}\frac{1}{\ep_0}\intl{0}{z'}\varrho(z'')\,\dd z''\cdot \underbrace{\ev_z\cdot\ev_z}_{=1}\dd z'
 }


 Válasszuk a potenciál referenciapontjának az $x$--$y$ síkot, vagyis rögzítsük az $U(z=0)=0$-t. A megoldás így:
 \al{ 
  U(z)
   &= -\frac{1}{\ep_0}\intl{0}{z}\intl{0}{z'}\varrho(z'')\,\dd z''\dd z'\;.
 }
 
\fi