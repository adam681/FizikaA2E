\feladat{10}{Legyen a térben a potenciál a következő:
 \begin{equation*}
  U(\rv)=\frac{1}{4\pi\varepsilon_{0}}\frac{\pv\rv}{r^{3}}
 \end{equation*}
 ahol $\pv$ konstans vektor, $\rv\neq\boldsymbol{0}$ és $r=|\rv|$. Mi lesz az elektromos térerősség az origón kívül?
}{}{}

\ifdefined\megoldas

 Megoldás: 

 A térerősség kiszámításához a feladatban megadott potenciálfüggvény gradiensét kell kiszámolni. A gradiensképzés során a függvényt az $x$, az $y$ és a $z$ koordináták szerint kell parciálisan deriválni. 

 Az integrál közvetlen kiszámítása helyett próbáljunk kicsit általánosabbak lenni. Mivel az $r=\sqrt{x^2+y^2+z^2}$ mennyiség valamilyen hatványát kell deriválnunk, ezért az $r^n$-t fogjuk deriválni. Vegyük észre, hogy a koordináták felcserélhető módon helyezkednek el a kifejezésben. Ha tudjuk pl.\ azt, hogy mi az $x$ szerinti derivált, akkor az $y$ szerinti derivált ugyanaz lesz, csak az $x$-et és az $y$-t fel kell cserélni. Deriváljunk az egyik (most az $x$) koordináta szerint:
 \al{
  \pder{r^n}{x}
   &=\pder{}{x}\big(\sqrt{x^2+y^2+z^2}\big)^n
    =\pder{}{x}\big(x^2+y^2+z^2\big)^\frac{n}{2}
  \\
   &=\frac{n}{2}\cdot\big(x^2+y^2+z^2\big)^{\frac{n}{2}-1}\cdot 2x
    =nx\cdot\big(x^2+y^2+z^2\big)^{\frac{n}{2}-1}\;.
  \\ 
   &=nx\cdot r^{n-2}\;.
 }

 Így tehát $r^n$ gradiense:
 \al{
  \grad r^n
   &=\der{r^n}{\rv}
    =\left(\pder{r^n}{x},\pder{r^n}{y},\pder{r^n}{z}\right)
    =n\left(x,y,z\right)r^{n-2}
    =n\cdot r^{n-2}\cdot \rv\;.
 } 

 A térerősség ebben a konkrét esetben:
 \al{
  \Ev(\rv)
   &=-\grad U(\rv)
    =-\frac{1}{4\pi\varepsilon_{0}}\grad \left(\frac{\pv\rv}{r^{3}}\right)
  \\
   &=-\frac{1}{4\pi\varepsilon_{0}}\left(\big(\grad (\pv\rv)\big)\frac{1}{r^{3}}+(\pv\rv)\grad \frac{1}{r^{3}}\right)\;,
 }
 ahol felhasználtuk a szorzatfüggvény deriválási szabályát. Az első tag:
 \al{
  \grad(\pv\rv)
   &=\left(\pder{}{x},\pder{}{y},\pder{}{z}\right)(p_x x+p_y y+p_z z)
    =(p_x,p_y,p_z)
    =\pv\;,
 }
 a másodikra pedig használhatjuk az előbb levezetett azonosságot:
 \al{
  \grad \frac{1}{r^{3}}
   &=\grad r^{-3}
    =-3\cdot r^{-5}\rv
    =-\frac{3}{r^{5}}\rv\;.
 }
 Így
 \al{
  \Ev(\rv)
   =\frac{1}{4\pi\varepsilon_{0}}\frac{3(\pv\rv)\rv-r^2\pv}{r^5}\;.
 }

 Megjegyzés: ez a potenciál és ez a térerősség a $\pv$ dipólmomentummal rendelkező dipólus tere.

\fi