\feladat{9}{
 A $0 \leq x \leq a$, $0 \leq y \leq a$ térrészben a potenciál
 \begin{equation*}
  U(x,y,z)=U_{0}\sin\left(\frac{\pi}{a}x\right)\left(\frac{1}{1-\mathrm{e}^{2\pi}}\mathrm{e}^{\frac{\pi}{a}y}+\frac{1}{1-\mathrm{e}^{-2\pi}}\mathrm{e}^{-\frac{\pi}{a}y}\right)
 \end{equation*}
 alakú, ahol $U_{0}$ és $a$ állandók. Mi ebben a térrészben a térerősség? Mennyi töltés van a térrészen belüli $0 \leq z \leq L$ négyzetes oszlopban?
}{}{}

\ifdefined\megoldas

 Megoldás: 

 A térerősség a potenciál negatív gradiense, vagyis
 \al{
  \Ev(x,y,z)=-\grad U(x,y,z)
   =-\left(\pder{U(x,y,z)}{x},\pder{U(x,y,z)}{y},\pder{U(x,y,z)}{z}\right)\;,
 }
 ahol
 \al{
  -\pder{U(x,y,z)}{x}
   &=U_{0}\frac{\pi}{a}\cos\left(\frac{\pi}{a}x\right)\left(\frac{1}{1-\mathrm{e}^{2\pi}}\mathrm{e}^{\frac{\pi}{a}y}+\frac{1}{1-\mathrm{e}^{-2\pi}}\mathrm{e}^{-\frac{\pi}{a}y}\right)
  \\
  -\pder{U(x,y,z)}{y}
   &=U_{0}\sin\left(\frac{\pi}{a}x\right)\left(\frac{1}{1-\mathrm{e}^{2\pi}}\frac{\pi}{a}\mathrm{e}^{\frac{\pi}{a}y}+\frac{1}{1-\mathrm{e}^{-2\pi}} \left(-\frac{\pi}{a}\right)\mathrm{e}^{-\frac{\pi}{a}y}\right)
  \\
   -\pder{U(x,y,z)}{z}
    &=0\;,
 }
 azaz
 \al{
  &\Ev(x,y,z)
  \\
   &\quad
    =-U_{0}\frac{\pi}{a} 
   \bigg[ 
    \cos\left(\frac{\pi}{a}x\right) \left(\frac{1}{1-\mathrm{e}^{2\pi}}\mathrm{e}^{\frac{\pi}{a}y}+\frac{1}{1-\mathrm{e}^{-2\pi}}\mathrm{e}^{-\frac{\pi}{a}y}\right),
  \\
   &\qquad\qquad\qquad\sin\left(\frac{\pi}{a}x\right)\left(\frac{1}{1-\mathrm{e}^{2\pi}}\mathrm{e}^{\frac{\pi}{a}y}-\frac{1}{1-\mathrm{e}^{-2\pi}}\mathrm{e}^{-\frac{\pi}{a}y}\right),
  \\
   &\qquad\qquad\qquad
    0\bigg]\;.    
 }

A térrészen belüli töltés kiszámolásához használjuk a Gauss-tételt. Ismerjük a téglatest határain a térerősséget, így a fluxust közvetlenül ki tudjuk számolni, amely a Gauss-tétel értelmében megegyezik a bent található töltés mennyiségével. 

\marginfigure{09fel_01fig}

Mivel a térerősségnek nincsen $z$ komponense, így a téglatest alsó és felső síkján nincsen fluxusjárulék, hiszen ott merőlegesek az $\mathbf{E}$ és a $\mathrm{d}\mathbf{A}$ vektorok. A többi négy oldalt az 
\al{ 
 (1)\colon\left\{
  \begin{array}{c}
   x=0 \\
   0<y<a \\
   0<z<L 
  \end{array}
 \right.
 &&
 (2)\colon\left\{
  \begin{array}{c}
   x=a \\
   0<y<a \\
   0<z<L 
  \end{array}
 \right.
 \\
 (3)\colon\left\{
  \begin{array}{c}
   0<x<a \\
   y=0 \\
   0<z<L 
  \end{array}
 \right.
 &&       
 (4)\colon\left\{
  \begin{array}{c}
   0<x<a \\
   y=a \\
   0<z<L 
  \end{array}
 \right.
}
egyenletek határozzák meg. 

Az első kettő lap esetén a felület normálisa $\mathbf{e}_x$ irányú, így a skaláris szorzás után csak az elektromos tér $x$ komponense marad meg. Ezen a két felületen az integrálás során az $x$ koordináta végig $0$, illetve az $a$ értéket veszi fel:
\begin{itemize}
 \item 
  Az (1)-es felületen: $\dd\Av_{(1)}=-\dd y\dd z \ev_x$, így:
  \begin{align}
   \Phi_{(1)}
    &=\intl{(1)}{} \Ev(x=0,y,z)\,\dd\Av
   \\
    &=
     \intl{0}{a}\intl{0}{L} -U_0 \frac{\pi}{a} \left(\frac{1}{1-\e^{2\pi}}\e^{\frac{\pi}{a}y} - \frac{1}{1-\e^{-2\pi}}\e^{-\frac{\pi}{a}y}\right) \ev_x (-\dd y\dd z \ev_x) 
   \\
    &=
     U_0 \frac{\pi}{a}\underbrace{\intl{0}{L}  \dd z}_{L}\intl{0}{a}\left(\frac{1}{1-\e^{2\pi}}\e^{\frac{\pi}{a}y} - \frac{1}{1-\e^{-2\pi}}\e^{-\frac{\pi}{a}y}\right)\dd y
   \\
    &=
     U_0 L\left(\frac{\e^\pi-1}{1-\e^{2\pi}} + \frac{\e^{-\pi}-1}{1-\e^{-2\pi}}\right)
     =
     -U_0 L
   \;. 
  \end{align}

 \item 
  A (2)-es felületen: $\dd\Av_{(2)}=\dd y\dd z \ev_x$, azaz
  \begin{align}
   \Phi_{(2)}
    &=\intl{(1)}{} \Ev(x=a,y,z)\,\dd\Av
   \\
    &=
     \intl{0}{a}\intl{0}{L} U_0 \frac{\pi}{a} \left(\frac{1}{1-\e^{2\pi}}\e^{\frac{\pi}{a}y} - \frac{1}{1-\e^{-2\pi}}\e^{-\frac{\pi}{a}y}\right) \ev_x (\dd y\dd z \ev_x) 
   \\
    &=
     -U_0 L
   \;. 
  \end{align}

 \item 
  A (3)-as felületen $\dd\Av_{(3)}=-\dd x\dd z \ev_y$, tehát:
  \begin{align}
   \Phi_{(3)}
    &=\int\mathbf{E}(x,y=0,z)\,\mathrm{d}\mathbf{A}
   \\
    &=
     \intl{0}{a}\intl{0}{L} -U_0 \frac{\pi}{a}\sin\left(\frac{\pi}{a}x\right) \left(\frac{1}{1-\mathrm{e}^{2\pi}} - \frac{1}{1-\mathrm{e}^{-2\pi}}\right) \mathbf{e}_y (-\mathrm{d}x\mathrm{d}z\mathbf{e}_y) 
   \\
    &=
     U_0 \frac{\pi}{a}\left(\frac{1}{1-\mathrm{e}^{2\pi}} - \frac{1}{1-\mathrm{e}^{-2\pi}}\right) \underbrace{\intl{0}{L}\mathrm{d}z}_{L} 
     \underbrace{\intl{0}{a}\sin\left(\frac{\pi}{a}x\right)\,\mathrm{d}x}_{2\frac{a}{\pi}} 
   \\
    &=
     2U_0L  \left(\frac{1}{1-\mathrm{e}^{2\pi}} - \frac{1}{1-\mathrm{e}^{-2\pi}}\right)\;. 
  \end{align}

 \item
  Az utolsó esetben $\dd\Av_{(4)}=\dd x\dd z \ev_y$, vagyis:
  \begin{align}
   \Phi_{(4)}
    &=
     \int \mathbf{E}(x,y=a,z)\,\dd\Av
   \\
    &=
     \intl{0}{a}\intl{0}{L} -U_0\frac{\pi}{a} \sin\left(\frac{\pi}{a}x\right) \left(\frac{\mathrm{e}^\pi}{1-\mathrm{e}^{2\pi}} - \frac{\mathrm{e}^{-\pi}}{1-\mathrm{e}^{-2\pi}}\right) \mathbf{e}_y (\mathrm{d}x\mathrm{d}z\mathbf{e}_y) 
   \\
    &=
     -U_0\frac{\pi}{a} \left(\frac{\mathrm{e}^\pi}{1-\mathrm{e}^{2\pi}} - \frac{\mathrm{e}^{-\pi}}{1-\mathrm{e}^{-2\pi}}\right) 
      \underbrace{\intl{0}{L}\,\mathrm{d}z}_{L} 
      \underbrace{\intl{0}{a}\sin\left(\frac{\pi}{a}x\right)\,\mathrm{d}x}_{2\frac{a}{\pi}} 
   \\
    &= 
     -2U_0L  \left(\frac{\mathrm{e}^\pi}{1-\mathrm{e}^{2\pi}} - \frac{\mathrm{e}^{-\pi}}{1-\mathrm{e}^{-2\pi}}\right)\;. 
  \end{align}

\end{itemize}

Ezek összege adja a teljes fluxust, amelyből az össztöltés:
\begin{align}
 Q_\textrm{össz}
  &=
    \varepsilon_0\Phi_\text{kocka}
   =\varepsilon_0 \left(\Phi_{(1)}+\Phi_{(2)}+\Phi_{(3)}+\Phi_{(4)}\right) 
 \\
  &=-2\ep_0 U_0 L 
    +2\ep_0 U_0 L\left(\frac{1-\mathrm{e}^\pi}{1-\mathrm{e}^{2\pi}}-\frac{1-\mathrm{e}^{-\pi}}{1-\mathrm{e}^{-2\pi}}\right)
 \\
  &=-2\ep_0 U_0 L 
    +2\ep_0 U_0 L\frac{-2\mathrm{e}^\pi-\mathrm{e}^{-2\pi}+2\mathrm{e}^{-\pi}+\mathrm{e}^{2\pi}}{2-\mathrm{e}^{2\pi}-\mathrm{e}^{-2\pi}}
\end{align}


\fi