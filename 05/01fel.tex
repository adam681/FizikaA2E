\feladat{1}{
 Mi a homogén $\Ev$ térerősség potenciálja?
}{}{}

\ifdefined\megoldas
  
 Megoldás: 

 A potenciál definíciója: $\Ev(x,y,z)=-\grad U(x,y,z)$, amely kifejtve a három komponensre:
 \begin{align}
  E_x(x,y,z)&=-\pder{U(x,y,z)}{x}\\
  E_y(x,y,z)&=-\pder{U(x,y,z)}{y}\\
  E_z(x,y,z)&=-\pder{U(x,y,z)}{z}\;,
 \end{align}
 ahol a $\partial$ a parciális deriválást jelenti. Homogén térerősség esetében $\Ev=(E_x,E_y,E_z)$. Tehát az $U$ függvénynek olyannak kell lennie, hogy $x$, $y$ és $z$ szerint is deriválva egy-egy konstanst kapunk. Azonnal látszik, hogy az $U(x,y,z)=-E_x\cdot x-E_y\cdot y-E_z\cdot z$ függvény jó választás, hiszen ez mind a három fenti egyenletet kielégíti:
 \al{
  -\grad U(x,y,z)
   &=-\left(\pder{}{x},\pder{}{y},\pder{}{z}\right)(-E_x\cdot x-E_y\cdot y-E_z\cdot z)
  \\
   &=(E_x,E_y,E_z)
    =\Ev\;.
 }

 Azonban vegyük észre azt is, a fent megadott potenciálfüggvényhez egy tetszőleges konstanst hozzáadhatok, a gradiens nem fog megváltozni, hiszen egy konstans deriváltja nulla. A lehető legáltalánosabb megoldás:
 \al{
  U(x,y,z)=-E_x\cdot x-E_y\cdot y-E_z\cdot z+C=-\Ev\rv+C\;.
 }
 
\fi