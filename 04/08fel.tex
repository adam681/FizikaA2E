\feladat{8}{
 $R = 1$\,cm sugarú végtelen hosszú körhenger homogén, $\varepsilon_{r}=5$ relatív permittivitású anyagból készült. A hengeren belül $\varrho=5/3\cdot10^{-3}\me{C/m}^{3}$ tértöltés, a hengeren kívül vákuum van. Mekkora a térerősség a tengelytől 0,5\,cm és 1,5\,cm távolságban?
}{}{}

\ifdefined\megoldas
 \marginfigure{08fel_01fig}
 
 Megoldás: 

 Az 5.~feladathoz hasonlóan itt is hengerszimmetrikus problémánk van.  A megfelelő Gauss-felület egy $l$ hosszú, a szigetelő hengerrel koaxiális henger alakú felület lesz. A henger sugara két tartományra eshet:
 \begin{description}
  \item [$r<R$:]
   Itt a Gauss-törvény egyik és másik oldala
   \al{
    \iiint\limits_\text{henger}\varrho(\rv)\,\dd^3\rv
     &=\iiint\limits_\text{henger}\varrho\,\dd^3\rv
      =\varrho\iiint\limits_\text{henger}\,\dd^3\rv
      =\varrho r^2\pi l\;,\\
    \iint\limits_\text{henger}\Dv(\rv)\,\dd^2\fv
     &=\iint\limits_\text{palást}\Dv(\rv)\,\dd^2\fv
      +2\underbrace{\iint\limits_\text{alaplap}\Dv(\rv)\,\dd^2\fv}_{=0}
      =\iint\limits_\text{henger}D(r)\,\dd^2 f\\
     &=D(r)\iint\limits_\text{henger}\,\dd^2 f
      =D(r)2r\pi l\;,
   }
   vagyis:
   \al{
    D(r)=\frac{\varrho}{2}r\;.
   }
   
  \item [$R<r$:]
   Ezen a tartományon a Gauss-felület az egész szigetelő hengert tartalmazza, vagyis $Q_\text{in}=\varrho \cdot R^2\pi l$. A felületi integrál az előző tartományhoz hasonlóan számítható, így
   \al{
    D(r)2r\pi l\
     &=\varrho\cdot R^2\pi l\\
    D(r)
     &=\frac{\varrho}{2}\frac{R^2}{r}\;.
   }
   
 \end{description}

 Összegfoglalva:
 \al{
  D(x)=
   \begin{cases}
     \frac{\varrho}{2}r & r<R \\
     \frac{\varrho}{2}\frac{R^2}{r} & R<r
   \end{cases}
 }
 ahonnan
 \al{
  E(x)=\frac{D(x)}{\ep_0\ep_\text{r}(x)}=
   \begin{cases}
     \frac{\varrho}{2\ep_0\ep_\text{r}}r & r<R \\
     \frac{\varrho}{2\ep_0}\frac{R^2}{r} & R<r
   \end{cases}
 }

 A megfelelő sugárértékek behelyettesítésével megkapjuk a feladat kérdésére a közvetlen választ. 

\fi