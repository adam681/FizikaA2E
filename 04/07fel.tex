\feladat{7}{
 $R_1=10\me{cm}$ sugarú gömb térfogati töltéssűrűsége $\varrho=300\me{C/m^{3}}$, relatív permittivitása $\varepsilon_{r}=5$. A gömböt körülveszi egy vele koncentrikus fém gömbhéj, amelynek sugarai $R_2=20\me{cm}$ és $R_3=23\me{cm}$. Ábrázolja a térerősség változását a középponttól mért távolság függvényében!
}{}{}

\ifdefined\megoldas

 Megoldás: 

 \marginfigure{07fel_01fig}
 A 4.~feladat gondolatmenetét itt is végigvezethetjük. A Gauss-gömb sugara itt az alábbi tartományokba eshet:
 \begin{description}
  \item [$r<R_1$:]
   Itt a Gauss-törvény egyik és másik oldala
   \al{
    \iiint\limits_\text{gömb}\varrho(\rv)\,\dd^3\rv
     &=\iiint\limits_\text{gömb}\varrho\,\dd^3\rv
      =\varrho\iiint\limits_\text{gömb}\,\dd^3\rv
      =\varrho\frac{4}{3}r^3\pi\;,\\
    \iint\limits_\text{gömb}\Dv(\rv)\,\dd^2\fv
     &=\iint\limits_\text{gömb}D(r)\,\dd^2 f
      =D(r)\iint\limits_\text{gömb}\,\dd^2 f\\
     &=D(r)4\pi r^2\;,
   }
   vagyis:
   \al{
    D(r)=\frac{\varrho}{3}r\;.
   }
   
  \item [$R_1<r<R_2$:]
   Ezen a tartományon a Gauss-felület az egész szigetelő gömböt tartalmazza, vagyis $Q_\text{in}=\varrho\frac{4}{3}R_1^3\pi(=Q)$. A felületi integrál az előző tartományhoz hasonlóan számítható, így
   \al{
    D(r)4\pi r^2\
     &=\varrho\frac{4}{3}R_1^3\pi\\
    D(r)
     &=\frac{\varrho}{3}\frac{R_1^3}{r^2}\;.
   }
 \end{description}

 \begin{description}
  \item [$R_2<r<R_3$:] 
   Ezen a tartományon a külső félgömbhéj belsejében vagyunk. Az ideális fém belsejében a térerősség és a dielektromos eltolás nulla. Ennek következménye az, hogy a gömbhéj belsejére $-Q$ töltésnek kell felhalmozódnia, hiszen a Gauss-törvény csak így teljesül. Mivel a fém nincs földelve, így a töltésmegmaradás miatt annak külsején $Q$ töltés halmozódik fel. 
   
  \item [$R_3<r$:]
   Itt is összesen $Q+Q-Q=Q$ töltés található a Gauss-felületen belül, vagyis a második tartományhoz hasonló megoldást kapjuk.
 \end{description}

 Összegfoglalva:
 \al{
  D(r)=
   \begin{cases}
     \frac{\varrho}{3}r & r<R_1 \\
     \frac{\varrho}{3}\frac{R_1^3}{r^2} & R_1<r<R_2 \\
     0 & R_2<r<R_3 \\
     \frac{\varrho}{3}\frac{R_1^3}{r^2} & R_3<r
   \end{cases}
 }
 ahonnan
 \al{
  E(r)=\frac{D(r)}{\ep_0\ep_\text{r}(r)}=
   \begin{cases}
     \frac{\varrho}{3\ep_0\ep_\text{r}}r & r<R_1 \\
     \frac{\varrho}{3\ep_0}\frac{R_1^3}{r^2} & R_1<r<R_2 \\
     0 & R_2<r<R_3 \\
     \frac{\varrho}{3\ep_0}\frac{R_1^3}{r^2} & R_3<r
   \end{cases}
 }

\fi