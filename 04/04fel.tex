\feladat{4}{
 $R_1=$10\,cm és $R_2=$20\,cm sugarú koncentrikus gömbök közötti teret $\varepsilon_{r}=3$ relatív permittivitású szigetelő tölti ki. A belső gömbre $Q$ töltést viszünk fel. Mekkora a szigetelőben a maximális térerősség? Hogyan változik az elektromos eltolás a középponttól való távolság függvényében?
}{04fel_01fig}{}

\ifdefined\megoldas
 
 Megoldás: 

 A Gauss-törvényt felhasználva fogjuk meghatározni az eltolásvektor sugárfüggését. A Gauss-törvény:
 \al{
  \iint\limits_{\partial V}\Dv(\rv)\,\dd^2\fv
   =\iiint\limits_{V}\rho(\rv)\,\dd^3\rv=Q_\text{in}\;,
 }
 ahol a bal oldalon a $\Dv$ dielektromos eltolás $V$ térfogat határára %felületére
 vett felületi integrálja áll, míg a jobb oldalon az elektromos töltéssűrűség $V$ térfogatra vett térfogati integrálja található. Ahhoz, hogy a fenti egyenletet ki tudjuk értékelni, egy megfelelő $V$ Gauss-térfogatot kell választani. Ahhoz pedig, hogy ezt ügyesen válasszunk meg, érdemes megnézni a rendszer szimmetriáit. 

 Értelemszerűen a rendszer gömbszimmetrikus. Ez súlyos következményekkel van arra vonatkozóan, hogy milyen alakú lehet a $\Dv(\rv)$ függvény.
 \begin{itemize}
  \item Ha gömbi koordináta-rendszerben gondolkodunk, akkor a $\Dv$ vektor az $r$, a $\vartheta$ és a $\varphi$ mennyiségektől függhet. A gömbszimmetria miatt azonban mindegy, hogy milyen szögnél nézzük a dielektromos eltolást, így $\Dv$ valójában csak $r$-től függhet.
  \item Emellett pedig a rendszernek minden olyan síkja tükörsík, amely átmegy a középponton. Ennek az a következménye, hogy a $\Dv$ csak sugárirányú lehet. Ez onnan látszik, hogy ha nem sugárirányú lenne, akkor van olyan tükörsík, amelyre nem esik rá ez a vektor. Ekkor viszont a vektor és a tükörképe nem ugyanaz. De ez ellentmondás, hiszen ha a rendszert tükrözzük a tükörsíkjára, akkor az eredeti rendszert kell látnunk.
 \end{itemize}

 Tehát valójában $\Dv(\rv)=D(r)\cdot\ev_\text{r}$, ahol $\ev_\text{r}$ a gömb középpontjától elfelé mutató egységvektor. 

 Ez alapján tehát érdemes egy olyan $r$ sugarú, gömb alakú Gauss-felületet választani, amelynek a közepe megegyezik a rendszer középpontjával. Ekkor ugyanis a Gauss-felület minden pontjára merőleges lesz a $\Dv$, illetve a felületen a $\Dv$ nagysága állandó lesz. 

 A Gauss-törvényt három tartományra tudjuk felírni. 
 \begin{description}
   \item[$r<R_1$:] Ekkor a Gauss-felület nem tartalmaz töltést ($Q_\text{in}=0$), így
    \al{
     0&=\iint\limits_{\partial V}\Dv(\rv)\,\dd^2\fv
       =\iint\limits_{\partial V}D(r)\,\dd^2 f
       =D(r)\iint\limits_{\partial V}\,\dd^2 f
       =D(r)4r^2\pi\\
     D(r)&=0
    }
   
   \item[$R_1<r<R_2$:] A Gauss-gömbön belül $Q$ töltés van, így
    \al{
     Q&=D(r)4r^2\pi
     &&\Rightarrow&
     D(r)&=\frac{1}{4\pi}\frac{Q}{r^2}\;.
    }
    
   \item[$R_2<r$:] A földelés miatt a külső fémgömbön felhalmozódott $-Q$ töltés, így a Gauss-gömbön belül $Q-Q=0$ töltés van. Ekkor szintén
    \al{
     D(r)&=0\;.
    }
 \end{description}

 Összefoglalva:
 \al{
  D(x)=
   \begin{cases}
     0 & r<R_1, \\
     \frac{1}{4\pi}\frac{Q}{r^2} & R_1<r<R_2, \\
     0 & R_2<r,
   \end{cases}
 }
 ahonnan
 \al{
  E(x)=\frac{D(x)}{\ep_0\ep_\text{r}(x)}=
   \begin{cases}
     0 & r<R_1, \\
     \frac{1}{4\pi\ep_0\ep_\text{r}}\frac{Q}{r^2} & R_1<r<R_2,\\
     0 & R_2<r.
   \end{cases}
 }
 
\fi