\feladat{3}{
 Egy origó középpontú, $R=10$\,cm sugarú, $m=1$\,kg tömegű homogén tömegeloszlású gömbre két erő hat: $\Fv_1=(2\iv+\jv)$\,N, amelynek a támadáspontja $\rv_1=0,1\kv$\,m, és $\Fv_2=(-2\iv-\jv)$\,N, amelynek a támadáspontja $\rv_2=-0,1\kv$\,m. Mi lesz a gömb szöggyorsulás-vektora? $t=1$\,s múlva milyen sebességű lesz az $\rv_1$ helyvektorú pontja?
}{03fel_01fig}{}
 
\ifdefined\megoldas 
  
 Megoldás: 

 A szöggyorsulás kiszámításához először meg kell határoznunk a gömbre ható forgatónyomatékot. Mivel a gömböt forgató két erő szimmetrikusan helyezkedik el, és ezek egymás ellentétei, ezért csak a tömegközéppont körüli forgómozgást fognak előidézni. Tudjuk, hogy ebben az esetben a forgástengely a gömb középpontján fog átmenni, így a forgatónyomatékot is erre vonatkozóan írjuk fel:
 \al{
  \Mv
   & = \rv_1\times\Fv_1 + \rv_2\times\Fv_2
     = 0,1\kv\times(2\iv+\jv) + (-0,1\kv)\times(-2\iv-\jv)\nonumber
  \\
   & = (-0,2\iv+0,4\jv)\me{\left[kg\frac{m^2}{s^2}\right]}\;.
 }

 A forgatónyomaték a szöggyorsulás-vektor és a tehetetlenségi nyomaték szorzata:
 \al{ 
  \Mv = \Theta \betav\;,
 }
 ahol a gömb tehetetlenségi nyomatéka $\Theta_\text{gömb}=\frac{2}{5}mR^2=\frac{2}{5}\cdot 1\me{kg}\cdot (0,1\me{m})^2=0,004\me{kg\,m^2}$. Behelyettesítve:
 \al{ 
  \betav 
   &= (-50\iv+100\jv)\me{\frac{1}{s^2}}\;.
 }

 Ha álló helyzetből indult a gömb, akkor a szögsebesség 1$\me{s}$ múlva:
 \al{
  \omv(t=1\me{s})
   &= \betav\cdot t
    =(-50\iv+100\jv)\me{\frac{1}{s}}\;,
 }
 a kerületi sebesség pedig:
 \al{ 
  \vv(1\me{s})
   &= \omv(t=1\me{s})\times \rv_1
    =(-50\iv+100\jv)\me{\frac{1}{s}}\times 0,1\kv\me{m}
    =(10\iv + 5\jv)\me{\frac{m}{s}}\;.
 }
 
\fi