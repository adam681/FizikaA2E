\feladat{10}{
 Határozzuk meg annak az $R$ sugarú gömbnek a tömegét és a $z$ tengelyre vonatkoztatott tehetetlenségi nyomatékát, amelynek a sűrűsége a $z$ tengely mentén változik: 
 \eq{
  \rho(x,y,z)=\rho_0+\alpha\cdot z\;.
 }
}{}{}

\ifdefined\megoldas
 
 Ahhoz, hogy meg tudjuk oldani a feladatot, tudnunk kell, hogy hogyan lehet tömeget és tehetetlenségi nyomatékot számolni. Ezeket az alábbi formulák adják meg:
 \al{
  &M=\iiint\limits_{\text{test}}\rho(x,y,z)\dd x\dd y\dd z\;,
  &
  &\Theta=\iiint\limits_{\text{test}}(x^2+y^2)\rho(x,y,z)\dd x\dd y\dd z\;,
 }
 ahol térfogati integrálokat kell kiszámítani. A térfogati integrál azt jelenti, hogy a teret felosztjuk elemi darabokra, Descartes-koordinátákban elemi $\dd x\dd y \dd z$ térfogatú téglatestekre, és összeadjuk az integrandus értékét megszorozva a térfogat nagyságával minden egyes pici téglatestre. Azonban Descartes-koordinátákban az integrált nagyon nehézkes lenne kiszámítani.
 
 Ehelyett a feladatot két különböző módon is megoldjuk, először hen\-ger\-ko\-or\-di\-ná\-ta-rend\-szer\-ben, másodszorra pedig gömbikoordináta-rendszerben. 
 
 {\textbf{ 1. megoldás:}} Mivel a sűrűség hengerszimmetrikus (a $z$ tengely körül forgásszimmetrikus), így talán érdemes bevezetni a hengerkoordinátákat. Az erre való áttérési szabály:
 \eq{
  \begin{cases}
   x=r\cdot \cos\varphi \\
   y=r\cdot \sin\varphi \\ 
   z=z
  \end{cases}
  \qquad \Leftrightarrow
  \qquad 
  \begin{cases}
   r=\sqrt{x^2+y^2} \\
   \varphi=\arctan{\left(\frac{y}{x}\right)} \\ 
   z=z
  \end{cases}
  \;.
 }
 
 \marginfigure[Hen\-ger\-ko\-or\-di\-ná\-ta-rend\-szer]{10fel_01fig.tikz}
 Az integrandus a tömeg esetében nem változik, a tehetetlenségi nyomatéknál pedig $(x^2+y^2)\rho(x,y,z)=r^2(\rho_0+\alpha\cdot z)$.
 
 Itt is meg kell vizsgálni, hogy a $\dd x\dd y\dd z$ térfogatelem helyett, mekkora lesz az infinitezimális térfogat. A polárkoordinátáknál láttuk, hogy egy $\dd r$ $\dd\varphi$ darabka területe $r\dd r\dd\varphi$, akkor egy ilyen alapú $\dd z$ magas térfogatelem térfogata $r\dd r\dd\varphi\dd z$. Vagyis
 \al{
  M
   &=\iiint\limits_{\text{test}}\rho(r,\varphi,z)\, r\dd r\dd \varphi\dd z
   =\iiint\limits_{\text{test}}(\rho_0+\alpha\cdot z)\, r\dd r\dd \varphi\dd z\;,\\
  \Theta
   &=\iiint\limits_{\text{test}}r^2(\rho_0+\alpha\cdot z)\, r\dd r\dd \varphi\dd z\;.
 }
 
 Nézzük meg, hogy hogyan tudjuk megadni az integrálási határokat, ha az origó középpontú $R$ sugarú gömböt akarjuk végigpásztázni. A $z$ koordináta $-R$-től $R$-ig megy. Mivel a rendszer forgásszimmetrikus, így a $\varphi$ $z$-től függetlenül minden esetben $0$ és $2\pi$ között vehet fel bármilyen értéket. Az $r$ koordináta viszont nem lehet akármi, annak határa függ attól, hogy milyen $z$ értéknél vagyunk. Könnyen végiggondolható, hogy $z$ magasságban a gömb metszete egy $\sqrt{R^2-z^2}$ sugarú kör, vagyis $r$ 0-tól eddig az értékig mehet. Ez leírva:
 \al{
  M
   &=\intl{0}{2\pi}\intl{-R}{R}\intl{0}{\sqrt{R^2-z^2}}(\rho_0+\alpha\cdot z)\, r\dd r\dd z\dd\varphi\nonumber
  \\
   &=\intl{0}{2\pi}\Bigg[\intl{-R}{R}\Bigg(\intl{0}{\sqrt{R^2-z^2}}(\rho_0+\alpha\cdot z)\, r\dd r\Bigg)\dd z\Bigg]\dd\varphi\;.
 }
 Itt mivel az $r$ szerinti integrál határa függ $z$-től, ezért ez a két integrál nem emelhető át egymáson. Azonban ezt is igen magától értetődően kell megoldani:
 \al{
  M
   &=\intl{0}{2\pi}\dd\varphi\intl{-R}{R}\Bigg(\intl{0}{\sqrt{R^2-z^2}}(\rho_0+\alpha\cdot z)\, r\dd r\Bigg)\dd z\nonumber
   \\
   &=2\pi \intl{-R}{R}\Bigg(\intl{0}{\sqrt{R^2-z^2}}(\rho_0+\alpha\cdot z)\, r\dd r\Bigg)\dd z\nonumber
   \\
   &=2\pi \intl{-R}{R}\Bigg(\intl{0}{\sqrt{R^2-z^2}} r\dd r\Bigg)(\rho_0+\alpha\cdot z)\,\dd z
    =2\pi \intl{-R}{R}\left[\frac{r^2}{2}\right]_{0}^{\sqrt{R^2-z^2}}(\rho_0+\alpha\cdot z)\,\dd z\nonumber
   \\
   &=\pi \intl{-R}{R}(R^2-z^2)(\rho_0+\alpha\cdot z)\,\dd z
    =\pi \intl{-R}{R}(R^2\rho_0+R^2\alpha\cdot z-z^2\rho_0-\alpha z^3)\,\dd z\nonumber
   \\
   &=\pi \left[R^2\rho_0\cdot z+R^2\alpha\cdot \frac{z^2}{2}-\frac{z^3}{3}\rho_0-\alpha \frac{z^4}{4}\right]_{-R}^{R}
    =2\pi \left(\rho_0-\frac{1}{3}\rho_0\right)R^3\nonumber
   \\
   &=\frac{4}{3}R^3\pi\cdot \rho_0\;.
 }
 A tehetetlenségi nyomaték ugyanígy:
 \al{
  \Theta
   &=\intl{0}{2\pi}\dd\varphi\intl{-R}{R}\intl{0}{\sqrt{R^2-z^2}}r^2\rho(x,y,z)\, r\dd r\dd z
    =2\pi\intl{-R}{R}\intl{0}{\sqrt{R^2-z^2}}r^3(\rho_0+\alpha\cdot z)\, \dd r\dd z\nonumber
   \\
   &=2\pi \intl{-R}{R}\left[\frac{r^4}{4}\right]_{0}^{\sqrt{R^2-z^2}}(\rho_0+\alpha\cdot z)\,\dd z\nonumber
    =\frac{\pi}{2} \intl{-R}{R}(R^2-z^2)^2(\rho_0+\alpha\cdot z)\,\dd z\nonumber
   \\
   &=\rho_0\frac{\pi}{2} \left[R^4z-2R^2\frac{z^3}{3}+\frac{z^5}{5}\right]_{-R}^{R}
    =R^5\rho_0 \pi \left(1-\frac{2}{3}+\frac{1}{5}\right)\nonumber
   \\
   &=\frac{8}{15}R^5\rho_0 \pi \frac{8}{15}
    =\frac{2}{5}\left(\frac{4}{3}R^3\pi \rho_0\right)R^2
    =\frac{2}{5}MR^2\;.
 }
 
 {\textbf{ 2. megoldás:}} Mivel a test határa gömbszimmetrikus ezért akár a gömbi koordinátákat is bevezethetjük. Az erre való áttérési szabály:
 \marginfigure[Gömb\-ko\-or\-di\-ná\-ta-rend\-szer]{10fel_02fig}
 \eq{
  \begin{cases}
   x=r\cdot \sin\vartheta\cos\varphi \\
   y=r\cdot \sin\vartheta\sin\varphi \\ 
   z=r\cdot \cos\vartheta
  \end{cases}
  \qquad \Leftrightarrow
  \qquad 
  \begin{cases}
   r=\sqrt{x^2+y^2+z^2} \\
   \vartheta=\arccos\left(\frac{z}{r}\right) \\
   \varphi=\arctan{\left(\frac{y}{x}\right)} 
  \end{cases}
  \;.
 }
 
 Levezethető, hogy az infinitezimális térfogat $\dd V=r^2\sin\vartheta\dd r\dd\vartheta\dd\varphi$. Ugyanazt kell tegyük, mint az előbb: be kell helyettesíteni az új koordinátákat a sűrűségfüggvénybe, meg kell határozni az integrálási határokat, majd el kell végezni az egyváltozós integrálokat. 
 
 Mivel egy gömbön belül integrálunk, így az $r$ változó $0\dots R$ tartományon lehet, a $\vartheta$ $0\dots\pi$-ig mehet, a $\varphi$ pedig $0\dots 2\pi$-ig. Így tehát
 
 \al{
  M
   &=\intl{0}{2\pi}\intl{0}{\pi}\intl{0}{R}(\rho_0+\alpha\cdot r\cos\vartheta)\, r^2\sin\vartheta\dd r\dd \vartheta\dd\varphi\nonumber
   \\
   &=2\pi\intl{0}{\pi}\intl{0}{R}(\rho_0+\alpha\cdot r\cos\vartheta)\, r^2\sin\vartheta\dd r\dd \vartheta\nonumber
   \\
   &=2\pi\intl{0}{\pi}\left(\rho_0\left[\frac{r^3}{3}\right]_{0}^{R}+\alpha\cdot \left[\frac{r^4}{4}\right]_{0}^{R}\cos\vartheta\right)\, \sin\vartheta\dd \vartheta\dd\nonumber
   \\
   &=\frac{2}{3}\rho_0R^3\pi\intl{0}{\pi}\sin\vartheta\dd \vartheta
    +2\pi\alpha\cdot\frac{R^4}{4}\intl{0}{\pi}\cos\vartheta\, \sin\vartheta\dd \vartheta\nonumber
   \\
   &=\frac{2}{3}\rho_0R^3\pi\left[-\cos\vartheta\right]_{0}^{\pi}
    +2\pi\alpha\cdot\frac{R^4}{4}\intl{0}{\pi}\frac{1}{2}\sin(2\vartheta)\dd \vartheta\nonumber
   \\
   &=\frac{4}{3}R^3\pi\cdot \rho_0
    +2\pi\alpha\cdot\frac{R^4}{4}\underbrace{\left[\frac{-\cos(2\vartheta)}{4}\right]_{0}^{\pi}}_{0}
    =\frac{4}{3}R^3\pi\cdot \rho_0\;.
 }
 és
 \al{
  \Theta
   &=\intl{0}{2\pi}\intl{0}{\pi}\intl{0}{R}r^2\sin^2\vartheta(\rho_0+\alpha\cdot r\cos\vartheta)\, r^2\sin\vartheta\dd r\dd \vartheta\dd\varphi\nonumber
   \\
   &=2\pi\intl{0}{\pi}\intl{0}{R}r^4\sin^3\vartheta(\rho_0+\alpha\cdot r\cos\vartheta)\,\dd r\dd \vartheta\nonumber
   \\
   &=2\pi\intl{0}{\pi}\left(\rho_0\left[\frac{r^5}{5}\right]_{0}^{R}+\alpha\cdot \left[\frac{r^6}{6}\right]_{0}^{R}\cos\vartheta\right)\, \sin^3\vartheta\dd \vartheta\nonumber
   \\
   &=\frac{2}{5}\rho_0R^5\pi\intl{0}{\pi}\sin^3\vartheta\dd \vartheta
    +2\pi\alpha\cdot\frac{R^6}{6}\intl{0}{\pi}\cos\vartheta\sin^3\vartheta\, \dd \vartheta\;.
 }
 
 Itt a második integrált könnyű kiszámítani, ha észrevesszük, hogy megjelent a $\sin^4\vartheta$ deriváltja:
 \al{
  \intl{0}{\pi}\cos\vartheta\sin^3\vartheta\, \dd \vartheta
   &=\left[\frac{\sin^4\vartheta}{4}\right]_{0}^{\pi}=0\;.
 }
 Az első integrál kicsit trükkösebb. Használjuk fel a trigonometrikus átalakító képleteket, hogy az előző trükköt ismét tudjuk használni:
 \al{
  \intl{0}{\pi}\sin^3\vartheta\,\dd \vartheta
   &=\intl{0}{\pi}\sin^2\cdot\sin\vartheta\,\dd \vartheta
    =\intl{0}{\pi}(1-\cos^2\vartheta)\cdot\sin\vartheta\,\dd\vartheta\nonumber
   \\
   &=\intl{0}{\pi}\sin\vartheta\,\dd\vartheta
    -\intl{0}{\pi}\cos^2\vartheta\sin\vartheta\,\dd\vartheta
    =\left[-\cos\vartheta\right]_{0}^{\pi}
    +\left[\frac{\cos^3\vartheta}{3}\right]_{0}^{\pi}\nonumber
   \\
   &=2-\frac{2}{3}
    =\frac{4}{3}\;,
 }
 tehát
 \al{
  \Theta
   =\frac{2}{5}\left(\frac{4}{3}R^3\pi \rho_0\right)R^2
   =\frac{2}{5}MR^2\;.
 }
 
 \emph{Megjegyzés:}
 
 Láthatjuk, hogy mind a két módszerrel ugyanazt a megoldást kaptuk, ez nem is lehetne másképp, az eredménynek minden esetben függetlennek kell lennie attól, hogy milyen koordináta-rendszert választunk. Vegyük észre, hogy a gömb tömege pont megegyezik egy $\rho_0$ sűrűségű homogén gömb tömegével. Ez magától értetődik hiszen ez egy olyan gömb, amelynek a teteje pont annyival nehezebb, mint amennyivel az alja könnyebb. A tehetetlenségi nyomatéka is megegyezik egy homogén gömbével. Ez is a szimmetria és a speciálisan megválasztott forgástengely következménye. Ha ezeket észrevesszük az elején, akkor egy kis indoklással egy sor integrálás nélkül is megadhatjuk az eredményt. 

\fi